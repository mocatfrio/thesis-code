\chapter{ANALISIS DAN PERANCANGAN SISTEM} \label{chapter:analisis dan perancangan sistem}
\tab Pada bab ini akan dijelaskan mengenai analisis dan perancangan sistem perangkat lunak yang akan dibangun, meliputi struktur data, algoritme, dan arsitektur aplikasi. 

\section{Daftar Istilah}
\tab Beberapa daftar istilah yang digunakan dalam bab ini dapat dilihat pada Tabel ? untuk melengkapi definisi yang telah dijelaskan pada Tabel \ref{tabel:daftar-simbol-bag1}.

\section{Analisis Sistem}
\tab Analisis sistem dibagi menjadi dua bagian, yaitu analisis permasalahan yang diangkat pada tugas akhir ini dan deskripsi umum sistem perangkat lunak yang akan dibangun.

\subsection{Analisis Permasalahan}
\tab Permasalahan yang diangkat dalam tugas akhir ini adalah pencarian $k$ produk yang paling menjanjikan pada interval waktu tertentu. Sebuah produk dikatakan "menjanjikan" jika ia memiliki nilai kontribusi pasar yang besar, dinilai dari tingkat kedekatannya dengan preferensi masing-masing pelanggan.

\subsection{Deskripsi Umum Sistem}
\tab 

\section{Perancangan Sistem}
\subsection{\textit{Data Preprocessing}}
\subsubsection{Struktur Data}
\subsubsection{Komputasi \textit{Dynamic Skyline}}
\subsection{Proses Utama}
\subsubsection{Pemrosesan Kueri \textit{k-Most Promising Products (k-MPP)}}
