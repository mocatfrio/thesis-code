\begin{appendices}

\chapter{Kode Sumber}
  \setcounter{figure}{0}
  \renewcommand{\thetable}{A.\arabic{table}}
  \renewcommand{\thefigure}{A.\arabic{figure}}

  \lstinputlisting[frame=single, language=scala, breaklines, caption=Kode sumber kelas \textit{Edge}, label=source:deletion, basicstyle=\tiny\ttfamily] {assets/code/ta/grid/Edge.scala}

  \lstinputlisting[frame=single, language=scala, breaklines, caption=Kode sumber kelas \textit{Node}, label=source:deletion, basicstyle=\tiny\ttfamily] {assets/code/ta/grid/Node.scala}

  \lstinputlisting[frame=single, language=scala, breaklines, caption=Kode sumber kelas \textit{Object}, label=source:deletion, basicstyle=\tiny\ttfamily] {assets/code/ta/grid/Object.scala}

  \lstinputlisting[frame=single, language=scala, breaklines, caption=Kode sumber pemrosesan utama, label=source:deletion, basicstyle=\tiny\ttfamily] {assets/code/ta/algorithm/TheAlgorithm.scala}

  \lstinputlisting[frame=single, language=scala, breaklines, caption=Kode sumber penentuan \textit{turning-point}, label=source:deletion, basicstyle=\tiny\ttfamily] {assets/code/ta/algorithm/TurningPoint.scala}

  \lstinputlisting[frame=single, language=scala, breaklines, caption=Kode sumber graf sementara, label=source:deletion, basicstyle=\tiny\ttfamily] {assets/code/ta/graph/TempGraph.scala}

  \lstinputlisting[frame=single, language=scala, breaklines, caption=Kode sumber penentuan grid, label=source:deletion, basicstyle=\tiny\ttfamily] {assets/code/ta/grid/Grid.scala}

  \lstinputlisting[frame=single, language=java, breaklines, caption=Kode sumber titik dua dimensi, label=source:deletion, basicstyle=\tiny\ttfamily] {assets/code/ta/geometry/Point2d.java}

  \lstinputlisting[frame=single, language=java, breaklines, caption=Kode sumber kotak dua dimensi, label=source:deletion, basicstyle=\tiny\ttfamily] {assets/code/ta/geometry/Rect2d.java}
  
  \lstinputlisting[frame=single, language=html, breaklines, caption=Kode sumber visualisasi HMTL, label=source:deletion, basicstyle=\tiny\ttfamily] {assets/code/visualisasi/main.html}
  
  \lstinputlisting[frame=single, breaklines, caption=Kode sumber visualisasi utama dengan JavaScript, label=source:deletion, basicstyle=\tiny\ttfamily] {assets/code/visualisasi/main.js}
  
  \lstinputlisting[frame=single, breaklines, caption=Kode sumber visualisasi dengan CSS, label=source:deletion, basicstyle=\tiny\ttfamily] {assets/code/visualisasi/style.css}
\end{appendices}