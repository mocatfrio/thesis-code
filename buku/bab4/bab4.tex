\chapter{IMPLEMENTASI} \label{chap:implementasi}

\tab Pada bab ini akan dijelaskan mengenai implementasi dari perancangan struktur data dan algoritme untuk menyelesaikan permasalahan \problemDua{} sebagaimana yang telah dijelaskan pada Bab \ref{chap:analisis-perancangan-sistem}. Penjelasan implementasi terdiri dari penjelasan kelas dan fungsi yang dibuat, disertai dengan \textit{pseudocode} untuk masing-masing fungsi tersebut.

\section{Lingkungan Implementasi}
\tab Lingkungan implementasi dalam pembuatan Tugas Akhir ini meliputi perangkat keras dan perangkat lunak dengan spesifikasi sebagai berikut:

\begin{enumerate}
	\item Perangkat Keras:
		\begin{itemize}
			\item Prosesor 2.6 GHz Intel Core i5 (I5-4278U)
			\item Memori 8 GB 1600 MHz DDR3
		\end{itemize}
	\item Perangkat Lunak:
	\begin{itemize}
		\item Sistem operasi macOS Mojave Versi 10.14.5
		\item \textit{Text editor} Visual Studio Code Versi 1.33.1
		\item Bahasa Pemrograman Python 3.7.3
		\item Flask \textit{Microframework} 1.0.2
	\end{itemize}			
\end{enumerate}

\section{Implementasi \textit{Data Precomputing}}
\tab \textit{Data precomputing} merupakan pemrosesan tahap pertama dalam algoritme $k$-MPPTI yang bertujuan untuk mengolah \textit{dataset} produk $P$ dan preferensi pelanggan $C$ dengan jumlah data sebanyak $n$ dan ukuran dimensi sebesar $d$, menjadi sebuah $Pandora Box$ yang menyimpan skor kontribusi pasar semua produk pada setiap waktu. Algoritme ini diimplementasikan menggunakan paradigma pemrograman berorientasi objek, sehingga semua data dan fungsi dibungkus dalam kelas-kelas. Ada empat macam kelas yang diimplementasikan, yaitu kelas \textit{EventQueue}, \textit{PandoraBox}, \textit{ReverseSkyline}, dan \textit{DynamicSkyline}.

\subsection{Kelas \textit{Event Queue}}
\tab Kelas \textit{EventQueue} mendefinisikan bentuk dan perilaku dari objek \textit{Event Queue} yang berfungsi untuk menyimpan \textit{event-event} yang terjadi di dalam himpunan data. Kelas ini mengimplementasikan karakteristik dari struktur data \textit{queue}, yakni FIFO (\textit{First In First Out}). 

\begin{figure}[H]
	\begin{algorithm}[H]
		\label{algo:event-queue-class}
		\caption{EventQueue Class}
		\begin{algorithmic}[1]
			\Statex \textbf{Input \hskip1em :} timestamp $t$, event's owner $o$ (product/customer), owner ID $oid$, action $a$ (insertion/deletion)
			\Statex \textbf{Output \hskip0.3em :} event queue $E$
			\Statex
			\State $E \gets$ empty list \Comment{Initialization}
			\Procedure{\large{E}\small{NQUEUE}}{$t$, $o$, $oid$, $a$}
				\State $e \gets$ list($t, o, oid, a$)
				\State $E \gets$ append $e$ 
			\EndProcedure
			\Procedure{\large{D}\small{EQUEUE}}{}
				\State $e \gets$ pop an element from $E$
				\State \Return $e$
			\EndProcedure
			\Procedure{\large{S}\small{ORT}\large{Q}\small{UEUE}}{}
				\State $E \gets$ sort elements in descending order based on sorting priority
			\EndProcedure
			\Procedure{\large{G}\small{ET}\large{T}\small{OTAL}\large{Q}\small{UEUE}}{}
				\State \Return number of elements in $E$
			\EndProcedure
			\Procedure{\large{G}\small{ET}\large{M}\small{AX}\large{T}\small{IMESTAMP}}{}
				\State \Return first index of $E$
			\EndProcedure
			\Procedure{\large{I}\small{S}\large{E}\small{MPTY}}{}
				\If {$E$ is empty} \Return True
				\Else \Return False
			\EndProcedure
		\end{algorithmic}
	\end{algorithm}
	\caption{Kelas \textit{EventQueue}}
\end{figure}
  
\subsection{Kelas \textit{Pandora Box}}

\begin{figure}[H]
	\begin{algorithm}[H]
		\label{algo:pbox-class}
		\caption{PandoraBox Class (Part 1)}
		\begin{algorithmic}[1]
			\Statex \textbf{Input \hskip1em :} $DSL(c)$, timestamp $ts$, probability score $Pr$ 
			\Statex \textbf{Output \hskip0.3em :} -
			\Statex
			\State $PBox \gets$ empty list  \Comment{Initialization}
			\Procedure{\large{A}\small{DD}\large{S}\small{CORE}}{$DSL(c)$, $ts$, $Pr$}
				\ForAll {$p\_id \in DSL(c)$}
					\State $last\_ts \gets 0$ 
					\State $last\_ts \gets$ last update of $DSL(c)$
					\If {$last\_ts > 0$} \Comment{update score first}
						\State call $UpdateScore(p\_id, last\_ts, ts, Pr)$
					\EndIf
					\If {$last\_ts \not= ts$} \Comment{then, add score}
						\State $PBox[p\_id][ts] \gets PBox[p\_id][ts] + Pr$  
						\State call $UpdateScore(p\_id, last\_ts, ts, Pr)$
					\EndIf
				\EndFor
			\EndProcedure
			\Procedure{\large{U}\small{PDATE}\large{S}\small{CORE}}{$p\_id$, $last\_ts$, $ts$, $Pr$}
				\For {$t \gets last\_ts+1$, $ts$}
					\State $PBox[p\_id][t] \gets PBox[p\_id][t] + Pr$  
				\EndFor
			\EndProcedure
			\State \Comment{Algo \ref{algo:pbox-class-2}}
			\algrestore{pbox}
		\end{algorithmic}
	\end{algorithm}
	\caption{Kelas \textit{PandoraBox} (Bagian 1)}
\end{figure}

\subsection{Kelas \textit{Reverse Skyline}}

\begin{figure}[H]
	\begin{algorithm}[H]
		\label{algo:rsl-class}
		\caption{ReverseSkyline Class}
		\begin{algorithmic}[1]
			\Statex \textbf{Input \hskip1em :} product as query point $p_s$, other products $p_o$, product dataset $P$, customer dataset $C$, maximum value $max\_val$
			\Statex \textbf{Output \hskip0.3em :} $RSL(p)$
			\Statex
			\State $d \gets$ len($max\_val$) \Comment{get dimension}
			\Procedure{\large{S}\small{TART}\large{C}\small{OMPUTATION}}{$p_s[id]$}
				\State call $DefineOrthant()$
				\State call $FindMidpointSkyline()$
				\State call $FindReveserSkyline()$
			\EndProcedure
			\Procedure{\large{D}\small{EFINE}\large{O}\small{RTHANT}}{}
				\State $O \gets$ empty dictionary
				\For {$i \gets 0$, $2^d$}
					\State $id \gets$ convert $i$ to binary
					\State $O[id] \gets$ empty dictionary
				\EndFor
			\EndProcedure
			\Procedure{\large{F}\small{IND}\large{M}\small{ID}\large{S}\small{KYLINE}}{}
				\ForAll {$p_o[id] \in P$}
					\If {$p_o[id]$ is $p_s[id]$ or $p_o[id]$ not in $P[active]$}
						\textbf{continue}
					\EndIf
					\State $m \gets$ call $CalcMidpoint(p_s[val], p_o[val])$ 
					\State $a \gets$ call $GetOrthantArea(p_o[val])$ 
					
				\EndFor
			\EndProcedure
			\Procedure{\large{G}\small{ET}\large{O}\small{RTHANT}\large{A}\small{REA}}{$p_o[val]$}
				\State $res \gets$ empty list
				\For {$i \gets 0$, $d$}
					\If {$p_o[val] \leq p_s[val]$}
						\State $res \gets$ append $0$
					\Else
						\State $res \gets$ append $1$
					\EndIf
				\EndFor
				\State \Return $res$
			\EndProcedure
			\Procedure{\large{C}\small{ALC\large{M}\small{IDPOINT}}{$val_1$, $val_2$} \Comment{calculate midpoint between products}
				\State $res \gets$ empty list
				\For {$i \gets 0$, $d$}
					\State $res \gets$ append $\frac{(val_1^i + val_2^i)}{2}$
				\EndFor
				\State \Return $res$
			\EndProcedure
		\end{algorithmic}
	\end{algorithm}
	\caption{Kelas \textit{ReverseSkyline}}
\end{figure}


\subsection{Kelas \textit{Dynamic Skyline}}

\subsection{Fungsi \textit{Main}}

\begin{figure}[H]
	\begin{algorithm}[H]
		\label{algo:main-func-1}
		\caption{MainFunction (List Event)}
		\begin{algorithmic}[1]
			\Statex \textbf{Input \hskip1em :} product dataset $P$, customer dataset $C$ 
			\Statex \textbf{Output \hskip0.3em :} pandora box $PB$
			\Statex
			\Statex \Comment{store data and event}
			\State $EQ \gets$ new $EventQueue()$ \Comment{EventQueue instantiation}
			\State $D \gets P, C$ 
			\ForAll {$d \in D$}
				\State open dataset file $d$
				\For {each row in $d$}
					\State $id \gets$ data ID
					\State $t_{in} \gets$ timestamp in
					\State $t_{out} \gets$ timestamp out
					\If {$d$ is product data} \Comment{product data}
						\State $dt \gets$ 0
					\Else \Comment{customer data}
						\State $dt \gets$ 1
					\EndIf
					\For {$i \gets$ 0, 2}
						\If {$i$ is 0} \Comment{insertion}
							\State call $EQ.Enqueue(t_{in}, dt, id, i)$
						\Else \Comment{deletion}
							\State call $EQ.Enqueue(t_{out}, dt, id, i)$
						\EndIf
					\EndFor
				\EndFor
			\EndFor
			\State call $EQ.SortQueue()$ \Comment{sorting queue}
			\Statex \Comment {break}
			\algstore{main1}
		\end{algorithmic}
	\end{algorithm}
\caption{Fungsi Utama (Mendaftar \textit{Event})}
\end{figure}
			
\begin{figure}[H]
	\begin{algorithm}[H]
		\label{algo:main-func-init}
		\caption{MainFunction (Inisialisasi)}
		\begin{algorithmic}[1]
			\algrestore{main1}
			\Statex \Comment {continued}
			\Statex \Comment{initialization} 
			\State $max\_ts \gets$ call $EQ.GetMaxTimestamp()$
			\State $max\_val \gets$ get max value of data
			\State $PB \gets$ new $PandoraBox(len(P), ts_{max})$  \Comment{PandoraBox instantiation}
			\State $P\_active \gets$ empty list of active product
			\State $C\_active \gets$ empty list of active customer
			\State $RSL \gets$ new $ReverseSkyline(P, C, max\_val)$  \Comment{ReverseSkyline instantiation}
			\State $DSL \gets$ new $DynamicSkyline(P, C, max\_val}, PB)$  \Comment{DynamicSkyline instantiation}
			\Statex \Comment break
			\algrestore{main2}
		\end{algorithmic}
	\end{algorithm}
\caption{Fungsi Utama (Inisialisasi)}
\end{figure}

\begin{figure}[H]
	\begin{algorithm}[H]
		\label{algo:main-func-event-process}
		\caption{MainFunction (Process Event)}
		\begin{algorithmic}[1]
			\algrestore{main2}
			\Statex \Comment {continued}
			\Statex \Comment{process event} 
			\While {$EQ.IsEmpty()$ is not True}
				\State $e \gets$ call $EQ.Dequeue()$
				\If {$e[o]$ is product}
					\If {$e[a]$ is insertion}
						\Comment{Algo \ref{algo:product-insertion}}
					\ElsIf {$e[a]$ is deletion}
						\Comment{Algo \ref{algo:product-deletion}}
					\EndIf
				\ElsIf{$e[o]$ is customer}
					\If {$e[a]$ is insertion}
						\Comment{Algo \ref{algo:customer-insertion}}
					\ElsIf {$e[a]$ is deletion}
						\Comment{Algo \ref{algo:customer-deletion}}
					\EndIf
				\EndIf
			\EndWhile
		\end{algorithmic}
	\end{algorithm}
	\caption{Fungsi Utama (Memproses \textit{Event})}
\end{figure}

\subsubsection{Fungsi \textit{Product Insertion}}
\begin{figure}[H]
	\begin{algorithm}[H]
		\label{algo:product-insertion}
		\caption{ProductInsertion}
		\begin{algorithmic}[1]
			\State $P\_active \gets$ append product id $e[oid]$
			\State $rsl\_result \gets $ call $RSL.StartComputation(e[oid])$
			\State $T \gets$ empty list of threads
			\For {each $r \in rsl\_rsesult$} \Comment{compute DSL}
				\State $th \gets$ make a thread that called $DSL.StartComputation(r, e[t], e[a], e[oid])$
				\State $T \gets$ append $th$
			\EndFor
			\State run threads
			\State join threads
			\State $T \gets$ empty list of threads
			\For {each $c \in Ca$} \Comment{update pandora box}
				\State $th \gets$ make a thread that called $DSL.StartComputation(c, e[t], update\_pbox)$
				\State $T \gets$ append $th$
			\EndFor
			\State run threads
			\State join threads
	\end{algorithmic} 
\end{algorithm}
\caption{Fungsi \textit{ProductInsertion}}
\end{figure}

\subsubsection{Fungsi \textit{Product Deletion}}
\begin{figure}[H]
	\begin{algorithm}[H]
		\label{algo:product-deletion}
		\caption{ProductDeletion}
		\begin{algorithmic}[1]
			\State $P\_active \gets$ append product id $e[oid]$
			\State $rsl\_result \gets $ call $RSL.StartComputation(e[oid])$
			\State $T \gets$ empty list of threads
			\For {each $r \in rsl\_rsesult$} \Comment{compute DSL}
			\State $th \gets$ make a thread that called $DSL.StartComputation(r, e[t], e[a], e[oid])$
			\State $T \gets$ append $th$
			\EndFor
			\State run threads
			\State join threads
			\State $T \gets$ empty list of threads
			\For {each $c \in Ca$} \Comment{update pandora box}
			\State $th \gets$ make a thread that called $DSL.StartComputation(c, e[t], update\_pbox)$
			\State $T \gets$ append $th$
			\EndFor
			\State run threads
			\State join threads
		\end{algorithmic} 
	\end{algorithm}
	\caption{Fungsi \textit{ProductDeletion}}
\end{figure}

\subsubsection{Fungsi \textit{Customer Insertion}}
\begin{figure}[H]
	\begin{algorithm}[H]
		\label{algo:customer-insertion}
		\caption{CustomerInsertion}
		\begin{algorithmic}[1]
			\State $C\_active \gets$ append customer id $e[oid]$ 
			\State call $DSL.StartComputation(e[oid], e[t], init\_dsl)$
			\State call $DSL.StartComputation(e[oid], e[t], update\_pbox)$
		\end{algorithmic} 
	\end{algorithm}
	\caption{Fungsi \textit{CustomerInsertion}}
\end{figure}

\subsubsection{Fungsi \textit{Customer Deletion}}
\begin{figure}[H]
	\begin{algorithm}[H]
		\label{algo:customer-deletion}
		\caption{CustomerDeletion}
		\begin{algorithmic}[1]
			\State call $DSL.StartComputation(e[oid], e[t], update\_pbox)$
			\State $C\_active \gets$ remove customer id $e[oid]$ 
		\end{algorithmic} 
	\end{algorithm}
	\caption{Fungsi \textit{CustomerDeletion}}
\end{figure}

\section{Implementasi Algoritme \textit{Query Processing}}

\subsection{Kelas \textit{Pandora Box}}

\begin{figure}[H]
	\begin{algorithm}[H]
		\label{algo:pbox-class-2}
		\caption{PandoraBox Class (Part 2)}
		\begin{algorithmic}[1]
			\algrestore{pbox}
			\Statex \textbf{Input \hskip1em :} product ID $p[id]$, time interval (time init $t_i$, time end $t_e$)
			\Statex \textbf{Output \hskip0.3em :} total market contribution score in time interval  $MC$
			\Statex 
			\Procedure{\large{G}\small{ET}\large{S}\small{CORE}}{$p[id]$, $t_i$, $t_e$} \Comment{get market contribution score}
				\State $MC \gets 0$ 
				\For {$i \gets t_i$, $t_e + 1$}
					\State $MC \gets MC + PBox[p[id]][i]$
				\EndFor
				\Return $MC$
			\EndProcedure
		\end{algorithmic}
	\end{algorithm}
	\caption{Kelas \textit{PandoraBox} (Bagian 2)}
\end{figure}

\section{Implementasi Algoritme \textit{Brute Force}}
\tab pending
\section{Implementasi Antarmuka Pengguna}
