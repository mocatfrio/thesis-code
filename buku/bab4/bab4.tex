\chapter{IMPLEMENTASI}
\label{chapter:implementasi}

Pada bab ini dijelaskan mengenai implementasi dari desain dan algoritma penyelesaian \problem{}.

\section{Lingkungan Implementasi}
Lingkungan implementasi dalam pembuatan Tugas Akhir ini meliputi perangkat keras dan perangkat lunak yang digunakan untuk penyelesaian \problem{} adalah sebagai berikut:

\begin{enumerate}
	\item Perangkat Keras:
		\begin{itemize}
			\item \textit{Processor} Intel(R) Core(TM) i3 - M 330 CPU @ 2.13GHz x 4 \item Memori 4 GB.
		\end{itemize}
	\item Perangkat Lunak:
	\begin{itemize}
		\item Sistem operasi Ubuntu 14.04 LTS 64 bit.
		\item \textit{Text editor} Sublime Text 3.
		\item \textit{Compiler} g++ versi 4.3.2.
	\end{itemize}			
\end{enumerate}


\section{Rancangan Data}
Pada subbab ini dijelaskan mengenai desain data masukan yang diperlukan untuk melakukan proses algoritma, dan data keluaran yang dihasilkan oleh program.

\subsection{Data Masukan}
Data masukan adalah data yang akan diproses oleh program sebagai masukan menggunakan algoritma dan struktur data yang telah dirancang dalam Tugas Akhir ini.\\
Data masukan berupa berkas teks yang berisi data dengan format yang telah ditentukan pada deskripsi \problem{}. Pada masing-masing berkas data masukan, baris pertama berupa sebuah bilangan bulat yang merepresentasikan jumlah kasus uji yang ada pada berkas tersebut. Untuk setiap kasus uji dengan tipe $1$ dan $2$, masukan berupa sebuah baris masukan yang terdiri dari dua buah parameter posisi berupa $x$ dan $y$. Sedangkan pada kasus uji dengan tipe $3$, masukan berupa sebuah parameter $y$ yang merupakan indeks \textit{string} yang dicari pada konfigurasi \textit{rope} saat ini.

\subsection{Data Keluaran}
Data keluaran yang dihasilkan oleh program hanya berupa satu nilai, yaitu huruf pada indeks ke-$y$ untuk setiap kasus uji dengan tipe $3$. 

\section{Implementasi Algoritma}
Pada subbab ini akan dijelaskan tentang implementasi proses algoritma secara keseluruhan berdasarkan desain yang telah dijelaskan pada Bab \ref{chapter:desain}.

\subsection{\textit{Header} yang Diperlukan}
Implementasi algoritma dengan pemanfaatan struktur data Rope untuk menyelesaikan \problem{} membutuhkan empat buah \textit{header} yaitu stdio.h, cstdlib, cstring dan utility, seperti yang terlihat pada Kode Sumber \ref{source:header}.\\

\lstinputlisting[language=C++, firstline=1, lastline=4, caption=\textit{Header} yang diperlukan, label=source:header]{assets/code/include.cpp}

\textit{Header} stdio.h berisi modul untuk menerima masukan dan memberikan keluaran. \textit{Header} cstdlib berisi modul untuk manajemen memori dinamis, generasi bilangan acak, pemilahan dan konversi. \textit{Header} cstring berisi modul yang memiliki fungsi-fungsi untuk melakukan pemrosesan \textit{string}. \textit{Header} utility mencakup berbagai modul yang menyediakan fungsionalitas mulai dari aplikasi perhitungan bit hingga aplikasi fungsi parsial. Contoh implementasinya penggunaan \textit{pair}. 

\subsection{Variabel Global}
Variabel global digunakan untuk memudahkan dalam mengakses data yang digunakan lintas fungsi. Kode sumber implementasi variabel global dapat dilihat pada Kode Sumber \ref{source:variabel_global}.

\begin{minipage}{\linewidth}
\lstinputlisting[language=C++, firstline=1, lastline=3, caption=Variabel Global, label=source:variabel_global]{assets/code/global.cpp}
\end{minipage}
	
\subsection{Implementasi Fungsi Main}
Fungsi Main adalah implementasi algoritma yang dirancang pada Gambar \ref{figure:fungsi_main}. Setiap tipe memiliki operasi yang berbeda-beda. Untuk operasi dengan tipe $3$, masukan hanya berupa parameter $y$ yang merupakan nilai posisi indeks yang dicari pada konfigurasi \textit{rope} saat ini. Pada operasi dengan tipe $1$ dan $2$, baris masukan berupa nilai $x$ dan $y$ yang merupakan posisi indeks dari \textit{string} pada \textit{rope}. Setiap masukan dengan tipe $3$ akan ditampilkan sebagai jawaban akhir dari permasalahan. Implementasi fungsi Main dapat dilihat pada Kode Sumber \ref{source:fungsi_main}.

\lstinputlisting[language=C++, firstline=1, lastline=24, caption=Fungsi Main, label=source:fungsi_main]{assets/code/main.cpp}

\subsection{Implementasi Struct Node}
Fungsi Struct Node berisi atribut yang dimiliki \textit{node} pada \textit{tree}. Implementasi dari fungsi Struct Node dapat dilihat pada Kode Sumber \ref{source:fungsi_struct}.

\lstinputlisting[language=C++, firstline=1, lastline=17, caption=Fungsi Struct Node, label=source:fungsi_struct]{assets/code/struct.cpp}

\subsection{Implementasi Fungsi Newnode}
Fungsi Newnode digunakan untuk membentuk sebuah \textit{node} yang berisikan karakter yang diberikan dan relasi terhadap karakter pada posisi yang bersebelahan. Sehingga membentuk sebuah \textit{tree} seperti pada Gambar \ref{fig:bst}. Implementasi fungsi \proc{newnode} dapat dilihat pada Kode Sumber \ref{source:fungsi_newnode}.

\lstinputlisting[language=C++, firstline=1, lastline=7, caption=Fungsi Newnode, label=source:fungsi_newnode]{assets/code/newnode.cpp}

\subsection{Implementasi Fungsi Build}
Fungsi Build membangun struktur \textit{tree} dari \textit{rope} yang dilakukan dari karakter yang berada pada posisi indeks di tengah sampai pada karakter paling awal maupun akhir.  Setiap \textit{node} akan memiliki anak kiri dan anak kanan jika dan hanya jika masih terdapat \textit{string} yang tersisa. Ilustrasinya dapat dilihat pada Gambar \ref{fig:rope}. Implementasi dari Fungsi Build dapat dilihat pada Kode Sumber \ref{source:fungsi_build}.

\lstinputlisting[language=C++, firstline=1, lastline=6, caption=Fungsi Build, label=source:fungsi_build]{assets/code/build.cpp}

\subsection{Implementasi Fungsi Getsize}
Fungsi Getsize digunakan untuk mendapatkan berat \textit{node} yang diperlukan. Implementasi dari fungsi Getsize dapat dilihat pada Kode Sumber \ref{source:fungsi_getsize}.

\lstinputlisting[language=C++, firstline=1, lastline=3, caption=Fungsi Getsize, label=source:fungsi_getsize]{assets/code/getsize.cpp}

\subsection{Implementasi Fungsi Split}
Fungsi Split memanfaatkan struktur data Pair yang berupa pasangan dari dua \textit{pointer} menuju \textit{node}. \textit{Pointer node} pertama akan berisi semua \textit{node} yang bernilai lebih kecil dari posisi indeks parameter \textit{split}, sedangkan \textit{pointer node} kedua berisi semua \textit{node} yang bernilai lebih besar sama dengan posisi indeks parameter \textit{split}.\\
Misalkan dilakukan \textit{split} pada indeks ke-$5$ dari \textit{rope} pada Gambar \ref{fig:split-5}, \textit{pointer node} pertama akan menyimpan semua \textit{node} dari indeks ke-$0$ sampai dengan $4$. Sedangkan \textit{pointer node} kedua menyimpan \textit{node} dari indeks ke-$5$ sampai akhir. Implementasi dari fungsi \textit{split} dapat dilihat pada Kode Sumber \ref{source:fungsi_split1} dan \ref{source:fungsi_split2}.
\begin{figure}
\centering
\begin{tikzpicture}[level/.style={sibling distance = 8cm/#1, level distance = 1.5cm}, scale=0.6,transform shape]
\node[treenode]{b,12}
  child{
    node[treenode]{t,6} 
    child {
    	node[treenode] {a,3}
    	child {
    		node[treenode]{g,1}
    	}
    	child{
    		node[treenode] {u,1}
    	}
    }
    child {
    	node[treenode] {m,2}
    	child {
    	    node[treenode] {a,1}
    	}
    	child[missing]{}
    }
  }
  child{
  	node[treenode]{h,5}
  	child {
  	    	node[treenode] {s,2}
  	    	child {
  	    		node[treenode]{i,1}
  	    	}
  	    	child[missing]{}
  	    }
  	    child {
  	    	node[treenode] {l,2}
  	    	child {
  	    	    node[treenode] {a,1}
  	    	}
  	    	child[missing] {}
  	    }
  };
  \draw [line width = 1pt] (-2.3,0) -- (-2.3,0) .. controls (-3.5,-2.5) .. (-2.1, -4.8);
\end{tikzpicture}
\begin{tikzpicture}[level/.style={sibling distance = 5cm/#1, level distance = 1.5cm}, scale=0.65,transform shape]
\node{
	\begin{tabular}{|c|c|}
		\hline
		pair first & pair second\\ \hline
	\end{tabular}
}
  child{
    node[treenode]{t,5} 
    child {
    	node[treenode] {a,3}
    	child {
    		node[treenode]{g,1}
    	}
    	child{
    		node[treenode] {u,1}
    	}
    }
    child {
    	node[treenode] {a,1}
    }
  }
  child{
  	node[treenode]{b,7}
  	child {
  		node[treenode]{m,1}
  	}
	child {
		node[treenode]{h,5}
		child {
			node[treenode] {s,2}
		  	child {
	  	   		node[treenode]{i,1}
  	    	}
  	    	child[missing]{}
  	    }
  	    child {
  	    	node[treenode] {l,2}
  	    	child {
  	    	    node[treenode] {a,1}
  	    	}
  	    	child[missing] {}
  	    }
	}
  };
\end{tikzpicture}
\caption{Ilustrasi Penyimpanan Hasil Operasi \textit{Split Rope} pada Struktur Data Pair \label{fig:split-5}}
\end{figure}

\lstinputlisting[language=C++, firstline=1, lastline=5, caption=Fungsi Split(1), label=source:fungsi_split1]{assets/code/split.cpp}
\lstinputlisting[language=C++, firstline=6, lastline=19, caption=Fungsi Split(2), label=source:fungsi_split2]{assets/code/split.cpp}

\subsection{Implementasi Fungsi Random}
Fungsi Random digunakan untuk menentukan posisi \textit{node} pada saat penggabungan dua buah \textit{rope}. Digunakan fungsi Random agar posisi \textit{node} tidak konstan dan berat suatu \textit{rope} tidak selalu sama. Implementasi fungsi Random ditunjukkan pada Kode Sumber \ref{source:fungsi_random}.

\lstinputlisting[language=C++, firstline=1, lastline=3, caption=Fungsi Random, label=source:fungsi_random]{assets/code/random.cpp}

\subsection{Implementasi Fungsi Concate}
Fungsi Concate menggabungkan dua buah \textit{rope} menjadi sebuah \textit{rope} utuh. Untuk setiap \textit{node} yang memiliki hasil modular nilai acak lebih kecil dari berat \textit{node root rope} pertama, akan menjadi \textit{root} dari \textit{rope} yang baru terbentuk. Untuk \textit{rope} kedua akan menjadi anak kanan atau anak kirinya, tergantung pada urutan BST yang seharusnya dibuat.\\
Misalkan terdapat dua buah \textit{rope} $R_1$ yang berisi \textit{string} "$abc$" dan $R_2$ yang berisi \textit{string} "$def$" seperti pada Gambar \ref{fig:concate-ab}. Apabila akan digabungkan dengan $R_1$ berada di posisi depan sehingga membentuk \textit{string} "$abcdef$", maka semua \textit{node} pada $R_2$ berada di posisi kanan rope $R_1$. Jika nilai hasil modular $R_1$ lebih kecil dari berat \textit{node root} $R_1$, \textit{node} tersebut akan menjadi \textit{root} dari \textit{rope} yang baru terbentuk. \textit{Node root} $R_2$ akan menjadi anak kanannya. Jika \textit{root} $R_1$ memiliki anak kanan, akan menjadi anak kiri dari \textit{node} $R_1$ yang memiliki indeks $0$. Implementasi Fungsi Concate ditunjukkan pada Kode Sumber \ref{source:fungsi_concate_1} dan \ref{source:fungsi_concate_2}.
\begin{figure}
\begin{tikzpicture}[level/.style={sibling distance = 5cm/#1, level distance = 1.5cm}, scale=0.65,transform shape]
\node[treenode]{b,3}
  child{
  	node[treenode]{a,1}
  }
  child{
  	node[treenode]{c,1}
  };
\end{tikzpicture}
\begin{tikzpicture}[level/.style={sibling distance = 5cm/#1, level distance = 1.5cm}, scale=0.65,transform shape]
\node[treenode]{e,3}
  child{
  	node[treenode]{d,1}
  }
  child{
  	node[treenode]{f,1}
  };
\end{tikzpicture}
\centerline
{\begin{tikzpicture}[level/.style={sibling distance = 5cm/#1, level distance = 1.5cm}, scale=0.7,transform shape]
\node[treenode]{b,6}
  child{
  	node[treenode]{a,1}
  }
  child{
  	node[treenode]{e,4}
  	child {
  		node[treenode] {d,2}
  		child {
  			node[treenode]{c,1}
  		}
  		child[missing]{}
  	}
  	child {
  		node[treenode] {f,1}
  	}
  };
\end{tikzpicture}}
\caption{Ilustrasi Operasi Concate. Setiap \textit{Node} Berisi Sebuah Karakter dan Nilai Berat Masing-Masing \textit{Node} \label{fig:concate-ab}}
\end{figure}

\lstinputlisting[language=C++, firstline=1, lastline=2, caption=Fungsi Concate(1), label=source:fungsi_concate_1]{assets/code/concate.cpp}

\lstinputlisting[language=C++, firstline=3, lastline=10, caption=Fungsi Concate(2), label=source:fungsi_concate_2]{assets/code/concate.cpp}

\subsection{Implementasi Fungsi Insert}
Fungsi Insert adalah implementasi dari desain algoritma pada Gambar \ref{figure:fungsi_insert}. Fungsi ini bertujuan untuk memasukkan data \textit{string} ke dalam \textit{rope}. Setiap \textit{string} akan dimasukkan ke dalam suatu \textit{node}. Setiap \textit{node} hanya berisi oleh satu karakter pecahan dari \textit{string} masukan. Implementasi dari Fungsi Insert dapat dilihat pada Kode Sumber \ref{source:fungsi_insert}.

\lstinputlisting[language=C++, firstline=1, lastline=4, caption=Fungsi Insert, label=source:fungsi_insert]{assets/code/insert.cpp}

\subsection{Implementasi Fungsi Mutable Begin}
Fungsi Mutable Begin adalah implementasi dari desain algoritma pada Gambar \ref{figure:fungsi_mutable_begin}. Operasi ini dilakukan untuk menjawab permasalahan pada subbab \ref{chapter:operasi1} dengan memanfaatkan dua buah operasi Split dan dua buah operasi Concate.\\
Misal dilakukan operasi $1$ $5$ $9$ pada \textit{rope} di Gambar \ref{fig:mutbegin}. Maka \textit{rope} akan dipotong pada posisi indeks ke-$5$ menghasilkan \textit{rope} $R_1$ dan $R_2$. Kemudian dilakukan \textit{split} kedua kali pada indeks $y-x+1$ pada \textit{rope} $R_2$ menghasilkan \textit{rope} $R_{21}$ dan $R_{22}$. Sehingga menghasilkan tiga buah \textit{rope} yang disimpan dalam struktur data Pair. Langkah selanjutnya dilakukan operasi Concate pada \textit{rope} $R_1$ dengan $R_{22}$. Hasil penggabungan \textit{rope} $R_1$ dengan $R_{22}$ akan digabungkan dengan \textit{rope} $R_{21}$. Pada operasi ini, \textit{rope} $R_{21}$ akan berada di posisi paling kiri dari keseluruhan \textit{rope}. Dan menghasilkan sebuah \textit{rope} utuh dengan urutan posisi \textit{rope} saat ini adalah $R_{21}$, $R_1$ dan $R_{22}$. Implementasi Fungsi Mutable Begin dapat dilihat pada Kode Sumber \ref{source:fungsi_mutable_begin}.
\begin{figure}
\centering
\begin{tikzpicture}[level/.style={sibling distance = 8cm/#1, level distance = 1.5cm}, scale=0.6,transform shape]
\node[splitnode]{b,12}
  child{
    node[treenode]{t,6} 
    child {
    	node[treenode] {a,3}
    	child {
    		node[treenode]{g,1}
    	}
    	child{
    		node[treenode] {u,1}
    	}
    }
    child {
    	node[splitnode] {m,2}
    	child {
    	    node[treenode] {a,1}
    	}
    	child[missing]{}
    }
  }
  child{
  	node[splitnode]{h,5}
  	child {
  	    	node[splitnode] {s,2}
  	    	child {
  	    		node[splitnode]{i,1}
  	    	}
  	    	child[missing]{}
  	    }
  	    child {
  	    	node[treenode] {l,2}
  	    	child {
  	    	    node[treenode] {a,1}
  	    	}
  	    	child[missing] {}
  	    }
  };
  \draw [line width = 1pt] (-2.3,0) -- (-2.3,0) .. controls (-3.5,-2.5) .. (-2.1, -4.8);
\end{tikzpicture}
\centering
\begin{tikzpicture}[level/.style={sibling distance = 5cm/#1, level distance = 1.5cm}, scale=0.68,transform shape]
\node[treenode]{t,5} 
    child {
    	node[treenode] {a,3}
    	child {
    		node[treenode]{g,1}
    	}
    	child{
    		node[treenode] {u,1}
    	}
    }
    child {
    	node[treenode] {a,1}
    };
\end{tikzpicture}
\centering
\begin{tikzpicture}[level/.style={sibling distance = 5cm/#1, level distance = 1.5cm}, scale=0.6,transform shape]
\node[splitnode]{b,7}
  child{
    node[splitnode]{m,1} 
  }
  child{
  	node[splitnode]{h,5}
  	child {
  	    	node[splitnode] {s,2}
  	    	child {
  	    		node[splitnode]{i,1}
  	    	}
  	    	child[missing]{}
  	    }
  	    child {
  	    	node[treenode] {l,2}
  	    	child {
  	    	    node[treenode] {a,1}
  	    	}
  	    	child[missing] {}
  	    }
  };
  \draw[-, line width = 1pt] (3.5,-1.8) -- (1.8,-4.5);
\end{tikzpicture}
\centering
\begin{tikzpicture}[level/.style={sibling distance = 5cm/#1, level distance = 1.5cm}, scale=0.6,transform shape]
\node[treenode]{t,7}
child {
	node[treenode]{a,3}
	child {
		node[treenode]{g,1}
	}
	child {
		node[treenode]{u,2}
	}
}
child {
	node[treenode]{l,3}
	child {
		node[treenode]{a,2}
		child {
			node[treenode]{a,1}
		}
		child[missing]{}
	}
	child[missing] {}
};
\end{tikzpicture}
\centering
\begin{tikzpicture}[level/.style={sibling distance = 3cm/#1, level distance = 1.5cm}, scale=0.55,transform shape]
\node[treenode]{b,5}
child {
	node[treenode]{m,1}
}
child{
	node[treenode]{h,3}
	child {
		node[treenode]{2,s}
		child {
			node[treenode]{i,1}
		}
		child[missing]{}
	}
	child[missing] {}
};
\end{tikzpicture}
\begin{tikzpicture}[level/.style={sibling distance = 5cm/#1, level distance = 1.5cm}, level 2/.style={sibling distance = 2cm}, level 3/.style={sibling distance = 1.5cm}, level 4/.style={sibling distance = 1.5cm}, scale=0.65,transform shape]
\node[treenode]{t,12}
child {
	node[treenode]{b,8}
	child {
		node[treenode]{m,1}
	}
	child {
		node[treenode]{h,6}
		child {
			node[treenode]{s,2}
			child {
				node[treenode]{i,1}
			}
			child[missing]{}
		}
		child {
			node[treenode]{a,3}
			child {
				node[treenode]{g,1}
			}
			child {
				node[treenode]{u,1}
			}
		}
	}
}
child {
	node[treenode]{l,3}
	child {
		node[treenode]{a,2}
		child {
			node[treenode]{a,1}
		}
		child[missing]{}
	}
	child[missing] {}
};
\end{tikzpicture}
\caption{Ilustrasi Operasi Mutable Begin. Setiap \textit{Node} Berisi Sebuah Karakter dan Nilai Prioritas Masing-Masing \textit{Node} \label{fig:mutbegin}}
\end{figure}

\lstinputlisting[language=C++, firstline=1, lastline=6, caption=Fungsi Mutable Begin, label=source:fungsi_mutable_begin]{assets/code/mutable_begin.cpp}

\subsection{Implementasi Fungsi Mutable End}
Fungsi Mutable End dilakukan untuk menjawab permasalahan pada subbab \ref{chapter:operasi2}. Implementasinya dilakukan dengan memanfaatkan dua buah operasi Split dan dua buah operasi Concate.\\
Misal dilakukan operasi $2$ $5$ $9$ pada rope di Gambar \ref{fig:mutend}. Maka \textit{rope} akan dipotong pada posisi indeks ke-$5$ menghasilkan rope $R_1$ dan $R_2$. Kemudian dilakukan \textit{split} kedua kali pada indeks $y-x+1$ pada \textit{rope} $R_2$ menghasilkan \textit{rope} $R_{21}$ dan $R_{22}$. Sehingga menghasilkan tiga buah \textit{rope} yang disimpan dalam struktur data Pair. Langkah selanjutnya dilakukan operasi Concate pada \textit{rope} $R_1$ dengan $R_{22}$. Hasil penggabungan \textit{rope} $R_1$ dengan $R_{22}$ akan digabungkan dengan \textit{rope} $R_{21}$. Pada operasi ini, \textit{rope} $R_{21}$ akan berada di posisi paling kanan dari keseluruhan \textit{rope}. Dan menghasilkan sebuah \textit{rope} utuh dengan urutan posisi \textit{rope} saat ini adalah $R_1$, $R_{22}$ dan $R_{21}$. Implementasi Fungsi Mutable End dapat dilihat pada Kode Sumber \ref{source:fungsi_mutable_end}.
\begin{figure}
\centering
\begin{tikzpicture}[level/.style={sibling distance = 8cm/#1, level distance = 1.5cm}, scale=0.6,transform shape]
\node[splitnode]{b,12}
  child{
    node[treenode]{t,6} 
    child {
    	node[treenode] {a,3}
    	child {
    		node[treenode]{g,1}
    	}
    	child{
    		node[treenode] {u,1}
    	}
    }
    child {
    	node[splitnode] {m,2}
    	child {
    	    node[treenode] {a,1}
    	}
    	child[missing]{}
    }
  }
  child{
  	node[splitnode]{h,5}
  	child {
  	    	node[splitnode] {s,2}
  	    	child {
  	    		node[splitnode]{i,1}
  	    	}
  	    	child[missing]{}
  	    }
  	    child {
  	    	node[treenode] {l,2}
  	    	child {
  	    	    node[treenode] {a,1}
  	    	}
  	    	child[missing] {}
  	    }
  };
  \draw [line width = 1pt] (-2.3,0) -- (-2.3,0) .. controls (-3.5,-2.5) .. (-2.1, -4.8);
\end{tikzpicture}
\centering
\begin{tikzpicture}[level/.style={sibling distance = 5cm/#1, level distance = 1.5cm}, scale=0.68,transform shape]
\node[treenode]{t,5} 
    child {
    	node[treenode] {a,3}
    	child {
    		node[treenode]{g,1}
    	}
    	child{
    		node[treenode] {u,1}
    	}
    }
    child {
    	node[treenode] {a,1}
    };
\end{tikzpicture}
\centering
\begin{tikzpicture}[level/.style={sibling distance = 5cm/#1, level distance = 1.5cm}, scale=0.6,transform shape]
\node[splitnode]{b,7}
  child{
    node[splitnode]{m,1} 
  }
  child{
  	node[splitnode]{h,5}
  	child {
  	    	node[splitnode] {s,2}
  	    	child {
  	    		node[splitnode]{i,1}
  	    	}
  	    	child[missing]{}
  	    }
  	    child {
  	    	node[treenode] {l,2}
  	    	child {
  	    	    node[treenode] {a,1}
  	    	}
  	    	child[missing] {}
  	    }
  };
  \draw[-, line width = 1pt] (3.5,-1.8) -- (1.8,-4.5);
\end{tikzpicture}
\centering
\begin{tikzpicture}[level/.style={sibling distance = 5cm/#1, level distance = 1.5cm}, scale=0.6,transform shape]
\node[treenode]{t,7}
child {
	node[treenode]{a,3}
	child {
		node[treenode]{g,1}
	}
	child {
		node[treenode]{u,2}
	}
}
child {
	node[treenode]{l,3}
	child {
		node[treenode]{a,2}
		child {
			node[treenode]{a,1}
		}
		child[missing]{}
	}
	child[missing] {}
};
\end{tikzpicture}
\centering
\begin{tikzpicture}[level/.style={sibling distance = 3cm/#1, level distance = 1.5cm}, scale=0.55,transform shape]
\node[treenode]{b,5}
child {
	node[treenode]{m,1}
}
child{
	node[treenode]{h,3}
	child {
		node[treenode]{2,s}
		child {
			node[treenode]{i,1}
		}
		child[missing]{}
	}
	child[missing] {}
};
\end{tikzpicture}
\begin{tikzpicture}[level/.style={sibling distance = 5cm/#1, level distance = 1.5cm}, level 2/.style={sibling distance = 3cm}, level 3/.style={sibling distance = 1.5cm}, level 4/.style={sibling distance = 1.5cm}, scale=0.65,transform shape]
\node[treenode]{b,12}
child {
	node[treenode]{t,8}
	child {
		node[treenode]{a,3}
		child {
			node[treenode]{g,1}
		}
		child {
			node[treenode]{u,1}
		}
	}
	child {
		node[treenode]{m,4}
		child {
			node[treenode]{l,3}
			child {
				node[treenode]{a,2}
				child {
					node[treenode]{a,1}
				}
				child[missing]{}
			}
			child[missing]{}
		}
		child[missing]{}
	}
}
child {
	node[treenode]{h,3}
	child {
		node[treenode]{s,2}
		child {
			node[treenode]{i,1}
		}
		child[missing]{}
	}
	child[missing] {}
};
\end{tikzpicture}
\caption{Ilustrasi Operasi Mutable End. Setiap \textit{Node} Berisi Sebuah Karakter dan Nilai Prioritas Masing-Masing \textit{Node} \label{fig:mutend}}
\end{figure}
\lstinputlisting[language=C++, firstline=1, lastline=6, caption=Fungsi Mutable End, label=source:fungsi_mutable_end]{assets/code/mutable_end.cpp}

\subsection{Implementasi Fungsi Print}
Fungsi Print digunakan untuk menjawab \problem{} sesuai dengan permasalahan pada subbab \ref{chapter:operasi3}. Implementasi fungsi ini berdasarkan dari desain algorithma pada Gambar \ref{figure:fungsi_print}. Implementasi dari Fungsi Print dapat dilihat pada Kode Sumber \ref{source:fungsi_print}.

\lstinputlisting[language=C++, firstline=1, lastline=6, caption=Fungsi Print, label=source:fungsi_print]{assets/code/print.cpp}