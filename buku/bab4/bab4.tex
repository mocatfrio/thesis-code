\chapter{IMPLEMENTASI} \label{chap:implementasi}

\tab Pada bab ini akan dijelaskan mengenai implementasi dari perancangan struktur data dan algoritme untuk menyelesaikan permasalahan \problemDua{} yang telah dijelaskan pada Bab \ref{chap:analisis-perancangan-sistem}. Penjelasan implementasi terdiri dari penjelasan kelas dan fungsi yang dibuat, disertai dengan \textit{pseudocode} untuk masing-masing fungsi tersebut.

\section{Lingkungan Implementasi}
\tab Lingkungan implementasi dalam pembuatan Tugas Akhir ini meliputi perangkat keras dan perangkat lunak dengan spesifikasi sebagai berikut:

\begin{enumerate}
	\item Perangkat Keras:
		\begin{itemize}
			\item Prosesor 2.6 GHz Intel Core i5 (I5-4278U)
			\item Memori 8 GB 1600 MHz DDR3
		\end{itemize}
	\item Perangkat Lunak:
	\begin{itemize}
		\item Sistem operasi macOS Mojave Versi 10.14.5
		\item \textit{Text editor} Visual Studio Code Versi 1.33.1
		\item Bahasa Pemrograman Python 3.7.3
		\item Flask \textit{Microframework} 1.0.2
	\end{itemize}			
\end{enumerate}

\begin{figure}[H]
	\begin{algorithm}[H]
		\label{algo:gsp}
		\caption{DetermineGSP}
		\begin{algorithmic}[1]
			\State \textbf{Input: }grid index \textit{G}, a distance $ d_\varepsilon $, uncertain data object $ X $, action(insertion/deletion)
			\State \textbf{Output: }an updated grid index \textit{G}
			\State create empty queue $ Q $
			\State create temporary graph $ Gr $
			\State access edge $ e $ enclosing $ X $ and enqueue $ n_i  $ and $ n_j $
			\State enqueue $ n_i $ and $ n_j $ with each distance
			\While{$ Q $ is not empty}
			\State sort Q by distance from $ X $
			\State dequeue $ Q $ as $ n $
			\If{$d_{n, X} \le d_\varepsilon $}
			\State insert grid enclosing $ n $ to $ Gr $
			\If{action is insertion}
			call $ Insertion() $
			\Else
			$ $ call $ Deletion() $
			\EndIf
			\ForAll{node $ m $ as neighbor of $ n $}
			\If{$ m $ has not visited}
			\State enqueue $ m $ with it's distance
			\State mark $ m $ as visited
			\EndIf
			\EndFor
			\EndIf
			\EndWhile
			\ForAll{edge $ e $ which has updated $ n_s $ or $ n_e $ in $ Gr $}
			\State find GSP as $ gsp $
			\State call \textit{ComputeTurningPoint($ gsp $)}
			\EndFor
		\end{algorithmic}
	\end{algorithm}
	\caption{Algoritme \textit{DetermineGSP}}
\end{figure}


\section{Implementasi Algoritme k-MPPTI}
\subsection{Implementasi Algoritme \textit{Data Precomputing}}
\subsubsection{\textit{Class}}
\subsection{Implementasi Algoritme \textit{Query Processing}}
\section{Implementasi Algoritme Pembanding (\textit{Brute Force})}
\section{Implementasi Antarmuka Pengguna}