% INDONESIAN ABSTRAK
\addcontentsline{toc}{chapter}{ABSTRAK}
\thispagestyle{plain}
\begin{centering}
\textbf{\MakeUppercase{\judul}}
\end{centering}

\begin{tabular}{ll}
Nama  & : \MakeUppercase{\penulis} \\
NRP & : \nrp \\
Departemen  & : \jurusan FTIK-ITS \\
Pembimbing I  & : \pembimbingSatu \\
Pembimbing II  & : \pembimbingDua
\end{tabular}
\\*[20pt]
\begin{centering}
\textbf{Abstrak}
\end{centering}
\itshape
% BEGIN
\\*[5pt]
Kemajuan ilmu pengetahuan dan teknologi telah mempengaruhi cara produsen dalam melakukan bisnis, yaitu dengan memanfaatkan data preferensi pelanggan untuk mendapatkan informasi, misalnya untuk mencari produk yang paling banyak diminati oleh pelanggan sehingga produsen dapat memilih produk dengan tepat untuk menarik lebih banyak pelanggan dan bertahan lama di pasar global.
\\*[5pt]
Saat ini, sudah ada komputasi yang dapat menyelesaikan permasalahan tersebut. \textit{k-Most Promising Products (k-MPP)} adalah sebuah strategi pemilihan produk dengan melakukan pencarian $k$-produk yang paling banyak diminati oleh pelanggan. Namun, hasil pencarian $k$-produk tidak mungkin tetap dari waktu ke waktu. Oleh karena itu, interval waktu adalah salah satu faktor yang harus dipertimbangkan dalam proses kueri karena sangat berpengaruh terhadap hasil pencarian. Permasalahan ini kemudian didefinisikan dalam penelitian ini sebagai "\textit{k-Most Promising Products} berbasis interval waktu (k-MPPTI)".
\\*[5pt]
Pada penelitian ini akan dirancang dan diimplementasikan struktur data dan algoritme yang tepat untuk menyelesaikan permasalahan tersebut. Data yang digunakan adalah data multidimensi, sehingga menggunakan struktur data yang memiliki fitur \textit{indexing}, yaitu \textit{array}. Algoritme \textit{k-MPPTI} menggunakan dua tipe kueri \textit{skyline}, yaitu \textit{dynamic skyline} dan \textit{reverse skyline}. Selain itu, algoritme k-MPPTI juga mengimplementasikan teknik komputasi paralel supaya pemrosesan data menjadi lebih cepat.
\\*[5pt]
Hasil uji coba menunjukkan bahwa algoritme k-MPPTI dapat memberikan hasil dengan waktu eksekusi ... kali lebih cepat dan penggunaan memori ... kali lebih hemat dibandingkan dengan algoritme konvensional Brute Force.
% END
\rm \\*[5pt]
\textbf{Kata Kunci: \textit{Strategi pemilihan produk, Kueri, Dynamic skyline, Reverse skyline, Interval waktu}}


\cleardoublepage

% ENGLISH ABSTRACT
\addcontentsline{toc}{chapter}{ABSTRACT}
\thispagestyle{plain}
\begin{centering}
\textbf{\MakeUppercase{\judulEnglish}}
\end{centering}

\begin{tabular}{ll}
Name  & : \MakeUppercase{\penulis} \\
NRP & : \nrp \\
Major  & : \jurusanEnglish Faculty of IT-ITS \\
Supervisor I  & : \pembimbingSatu \\
Supervisor II  & : \pembimbingDua
\end{tabular}
\\*[20pt]
\begin{centering}
\textbf{Abstract}
\end{centering}
\itshape
% BEGIN
\\*[5pt]
Kemajuan ilmu pengetahuan dan teknologi telah mempengaruhi cara produsen dalam melakukan bisnis, yaitu dengan memanfaatkan data preferensi pelanggan untuk mendapatkan informasi, misalnya untuk mencari produk yang paling banyak diminati oleh pelanggan sehingga produsen dapat memilih produk dengan tepat untuk menarik lebih banyak pelanggan dan bertahan lama di pasar global.
\\*[5pt]
Saat ini, sudah ada komputasi yang dapat menyelesaikan permasalahan tersebut. \textit{k-Most Promising Products (k-MPP)} adalah sebuah strategi pemilihan produk dengan melakukan pencarian $k$-produk yang paling banyak diminati oleh pelanggan. Namun, hasil pencarian $k$-produk tidak mungkin tetap dari waktu ke waktu. Oleh karena itu, interval waktu adalah salah satu faktor yang harus dipertimbangkan dalam proses kueri karena sangat berpengaruh terhadap hasil pencarian. Permasalahan ini kemudian didefinisikan dalam penelitian ini sebagai "\textit{k-Most Promising Products} berbasis interval waktu (k-MPPTI)".
\\*[5pt]
Pada penelitian ini akan dirancang dan diimplementasikan struktur data dan algoritme yang tepat untuk menyelesaikan permasalahan tersebut. Data yang digunakan adalah data multidimensi, sehingga menggunakan struktur data yang memiliki fitur \textit{indexing}, yaitu \textit{array}. Algoritme \textit{k-MPPTI} menggunakan dua tipe kueri \textit{skyline}, yaitu \textit{dynamic skyline} dan \textit{reverse skyline}. Selain itu, algoritme k-MPPTI juga mengimplementasikan teknik komputasi paralel supaya pemrosesan data menjadi lebih cepat.
\\*[5pt]
Hasil uji coba menunjukkan bahwa algoritme k-MPPTI dapat memberikan hasil dengan waktu eksekusi ... kali lebih cepat dan penggunaan memori ... kali lebih hemat dibandingkan dengan algoritme konvensional Brute Force.
% END
\rm \\*[5pt]
\textbf{Keywords: \textit{Product selection strategy, Query, Dynamic skyline, Reverse skyline, Time interval}}

\cleardoublepage
