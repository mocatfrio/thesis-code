% INDONESIAN ABSTRAK
\addcontentsline{toc}{chapter}{ABSTRAK}
\thispagestyle{plain}
\begin{centering}
\textbf{\MakeUppercase{\judul}}
\end{centering}

\begin{tabular}{ll}
Nama  & : \MakeUppercase{\penulis} \\
NRP & : \nrp \\
Departemen  & : \jurusan FTIK-ITS \\
Pembimbing I  & : \pembimbingSatu \\
Pembimbing II  & : \pembimbingDua
\end{tabular}
\\*[20pt]
\begin{centering}
\textbf{Abstrak}
\end{centering}
\itshape
% BEGIN
\\*[5pt]
Kemajuan ilmu pengetahuan dan teknologi di bidang analisis data telah mempengaruhi cara perusahaan dalam berbisnis, yakni dengan mengumpulkan data penjualan dan preferensi pelanggan, kemudian memanfaatkannya untuk mendapatkan informasi yang dapat digunakan untuk membuat keputusan bisnis yang tepat. 
\\*[5pt]
Saat ini, sudah ada strategi pemilihan produk dengan melakukan pencarian $k$-produk yang paling banyak diminati oleh pelanggan, yaitu \textit{k-Most Promising Products (k-MPP)}. Komputasi k-MPP menggunakan dua tipe kueri \textit{skyline}, yaitu \textit{dynamic skyline} dan \textit{reverse skyline}. Kelemahan dari komputasi k-MPP adalah tidak mempertimbangkan variabel waktu, sehingga hasil kuerinya tidak valid untuk digunakan sebagai bahan pertimbangan pembuat keputusan. Selain itu, komputasi k-MPP tidak dapat memproses kueri berbasis interval waktu.
\\*[5pt]
Tugas Akhir ini bertujuan untuk menjawab permasalahan k-MPP berbasis interval waktu pada data multidimensi dengan serial waktu dengan memodelkan kueri k-MPPTI \textit{(k-Most Promising Products in Time Intervals)}, serta merancang kerangka kerja algoritme yang dapat memproses kueri tersebut. Algoritme diimplementasikan menggunakan teknik komputasi paralel supaya pemrosesan data menjadi lebih cepat. Efektivitas dan efisiensi algoritme diuji menggunakan data asli dan sintetis.
\\*[5pt]
Hasil uji coba menunjukkan bahwa algoritme k-MPPTI yang tidak menggunakan komputasi \textit{reverse skyline} dapat memberikan hasil dengan waktu eksekusi ... kali lebih cepat dan penggunaan memori ... kali lebih hemat dibandingkan dengan algoritme k-MPPTI yang menggunakan komputasi \textit{reverse skyline}. Selain itu ...
% END
\rm \\*[5pt]
\textbf{Kata Kunci: \textit{Strategi Pemilihan Produk, Kueri, Dynamic Skyline, Reverse Skyline, Interval Waktu}}


\cleardoublepage

% ENGLISH ABSTRACT
\addcontentsline{toc}{chapter}{ABSTRACT}
\thispagestyle{plain}
\begin{centering}
\textbf{\MakeUppercase{\judulEnglish}}
\end{centering}

\begin{tabular}{ll}
Name  & : \MakeUppercase{\penulis} \\
NRP & : \nrp \\
Major  & : \jurusanEnglish Faculty of IT-ITS \\
Supervisor I  & : \pembimbingSatu \\
Supervisor II  & : \pembimbingDua
\end{tabular}
\\*[20pt]
\begin{centering}
\textbf{Abstract}
\end{centering}
\itshape
% BEGIN
\\*[5pt]
The advancement of science and technology, especially in the data analytics area, has influenced the way manufacturers do businesses by collecting customer preferences and sales data, then using it to obtain some informations to make the right business decision.
\\*[5pt]
Currently, there is a product selection strategy by searching for k-most preferred product by customers, namely k-Most Promising Products (k-MPP). This computation uses two types of skyline queries, dynamic skyline and reverse skyline. Unfortunately, k-MPP computation has some shortcomings. First, it doesn't consider the time variable, so the query results are invalid to be used as a consideration in decision making based on time. Second, k-MPP computing cannot process query based on time intervals.
\\*[5pt]
This study aims to answer the k-MPP query based on time intervals in multidimensional time series data with serial time by modeling k-Most Promising Products in Time Intervals (k-MPPTI) query and designing an algorithmic framework for processing the query. The algorithm is implemented using parallel computing techniques to make data processing faster. The effectiveness and efficiency of the algorithm was tested using real and synthetic datasets.
\\*[5pt]
Hasil uji coba menunjukkan bahwa algoritme k-MPPTI dapat memberikan hasil dengan waktu eksekusi ... kali lebih cepat dan penggunaan memori ... kali lebih hemat dibandingkan dengan algoritme konvensional Brute Force.
% END
\rm \\*[5pt]
\textbf{Keywords: \textit{Product Selection Strategy, Query, Dynamic Skyline, Reverse Skyline, Time Interval}}

\cleardoublepage
