% INDONESIAN ABSTRAK
\addcontentsline{toc}{chapter}{ABSTRAK}
\thispagestyle{plain}
\begin{centering}
\centering
\textbf{\MakeUppercase{\judul}}
\end{centering}

\begin{tabular}{ll}
Nama  & : \MakeUppercase{\penulis} \\
NRP & : \nrp \\
Departemen  & : \jurusan FTIK-ITS \\
Pembimbing I  & : \pembimbingSatu \\
Pembimbing II  & : \pembimbingDua
\end{tabular}
\\*[20pt]
\begin{centering}
\textbf{Abstrak}
\end{centering}
\itshape
% BEGIN
\\*[5pt]
Kemajuan ilmu pengetahuan dan teknologi, terutama di bidang analisis data, telah mempengaruhi cara perusahaan dalam menjalankan bisnis, yaitu dengan mengumpulkan data preferensi pelanggan dari data penjualan produk, kemudian memanfaatkannya untuk mendapatkan informasi yang dapat digunakan untuk membuat keputusan bisnis yang tepat. Saat ini, sudah ada penelitian yang mengembangkan strategi pemilihan produk dengan melakukan pencarian $k$-produk yang paling banyak diminati oleh pelanggan bernama \textit{k-Most Promising Products} (k-MPP). Komputasi k-MPP menggunakan dua tipe kueri \textit{skyline}, yaitu \textit{dynamic skyline} dan \textit{reverse skyline}. Sayangnya, komputasi k-MPP tidak mempertimbangkan variabel waktu dalam algoritme perhitungannya dan tidak dapat digunakan untuk memproses kueri berbasis interval waktu.
\\*[5pt]
Tugas Akhir ini bertujuan untuk menjawab permasalahan k-MPP berbasis interval waktu pada data multidimensi dengan serial waktu dengan memodelkan kueri k-MPPTI \textit{(k-Most Promising Products in Time Intervals)} dan merancang kerangka kerja algoritme yang dapat memproses kueri tersebut. Ada tiga jenis algoritme yang dibuat dan dibandingkan, yaitu k-MPPTI (menggunakan kueri \textit{dynamic skyline} dan \textit{reverse skyline}), k-MPPTI NoRSL (menggunakan kueri \textit{dynamic skyline} saja), dan k-MPPTI NoRSL-P (menggunakan teknik komputasi paralel). Efektivitas dan efisiensi algoritme diuji menggunakan data asli dan sintetis.
\\*[5pt]
Hasil uji coba menunjukkan bahwa algoritme k-MPPTI NoRSL memiliki performa yang lebih baik daripada algoritme k-MPPTI karena dapat memberikan hasil kueri dengan waktu eksekusi lima kali lebih cepat dan penggunaan memori satu kali lebih hemat dibandingkan dengan algoritme k-MPPTI.
% END
\rm \\*[5pt]
\textbf{Kata Kunci: \textit{Strategi Pemilihan Produk, Kueri, Dynamic Skyline, Reverse Skyline, Interval Waktu}}


\cleardoublepage

% ENGLISH ABSTRACT
\addcontentsline{toc}{chapter}{ABSTRACT}
\thispagestyle{plain}
\begin{centering}
\textbf{\MakeUppercase{\judulEnglish}}
\end{centering}

\begin{tabular}{ll}
Name  & : \MakeUppercase{\penulis} \\
NRP & : \nrp \\
Major  & : \jurusanEnglish Faculty of IT-ITS \\
Supervisor I  & : \pembimbingSatu \\
Supervisor II  & : \pembimbingDua
\end{tabular}
\\*[20pt]
\begin{centering}
\textbf{Abstract}
\end{centering}
\itshape
% BEGIN
\\*[5pt]
The advancement of science and technology, especially in the data analytics area, has influenced the way manufacturers do businesses by collecting customer preferences from product sales data, then using it to obtain some informations to make the right business decision. Currently, there is a product selection strategy by searching for k-most preferred product by customers, namely k-Most Promising Products (k-MPP). This computation uses two types of skyline queries, dynamic skyline and reverse skyline. Unfortunately, k-MPP computation doesn't consider the time variable and can't process query based on time intervals.
\\*[5pt]
This study aims to answer the k-MPP query based on time intervals in multidimensional time series data with serial time by modeling k-Most Promising Products in Time Intervals (k-MPPTI) query and designing an algorithmic framework for processing the query. There are three types of algorithm built and compared namely k-MPPTI (using both dynamic skyline and reverse skyline queries), k-MPPTI NoRSL (only using dynamic skyline), and k-MPPTI NoRSL-P (using parallel computing techniques). The effectiveness and efficiency of the algorithm was tested using real and synthetic datasets.
\\*[5pt]
\\
Based on the testing results, k-MPPTI NoRSL algorithm has better performance than k-MPPTI algorithm because it provides query results with execution time five times faster and memory usage one-time more efficient than k-MPPTI algorithm.
% END
\rm \\*[5pt]
\textbf{Keywords: \textit{Product Selection Strategy, Query, Dynamic Skyline, Reverse Skyline, Time Interval}}

\cleardoublepage
