\chapter{PENDAHULUAN}
\tab Pada bab ini dijelaskan latar belakang, rumusan masalah, batasan masalah, tujuan, manfaat, metodologi dan sistematika penulisan Tugas Akhir.

\section{Latar Belakang}
\tab Pesatnya kemajuan ilmu pengetahuan dan teknologi di bidang analisis data telah mempengaruhi cara perusahaan dalam menjalankan bisnis, yaitu dengan mengumpulkan data-data penjualan, riset pasar, logistik, atau biaya transportasi, kemudian menggunakannya untuk membuat keputusan bisnis yang lebih baik. Seorang analis dapat mengumpulkan data preferensi pelanggan terhadap fitur-fitur produk perusahaan tersebut dari data penjualan yang dimiliki. Selain itu, maraknya penggunaan situs web untuk menjual produk secara \textit{online} juga memungkinkan analis mengumpulkan data preferensi pelanggan terhadap fitur produk perusahaan lain.

Dengan memanfaatkan data preferensi pelanggan, sebuah perusahaan dapat mendapatkan informasi yang dapat digunakan untuk membuat keputusan bisnis yang tepat. Misalnya, dengan mendapatkan informasi $k$-produk apa saja yang paling diminati oleh pelanggan beserta fitur-fiturnya, perusahaan dapat menentukan harga produk baru yang akan diluncurkan atau menentukan fitur apa yang hendak diunggulkan dari produk baru yang ingin dibuat. 

\pagebreak
Saat ini, sudah ada penelitian yang mengembangkan strategi pemilihan produk dengan melakukan pencarian $k$-produk yang paling banyak diminati oleh pelanggan. Dalam \cite{kmpp}, Islam et al. memodelkannya sebagai kueri \textit{k-Most Promising Products} (k-MPP) dan membuat kerangka kerja algoritme untuk memproses kueri tersebut. Komputasi k-MPP menggunakan dua tipe kueri \textit{skyline}, yaitu \textit{dynamic skyline} \cite{dynamic-skyline} dan \textit{reverse skyline} \cite{reverse-skyline}. Kueri \textit{dynamic skyline} digunakan untuk mengambil data produk terbaik berdasarkan sudut pandang pelanggan, sedangkan kueri \textit{reverse skyline} digunakan untuk mengambil data pelanggan potensial berdasarkan sudut pandang produk atau perusahaan.

Kelemahan dari strategi pemilihan produk k-MPP adalah tidak adanya pertimbangan variabel waktu dalam algoritme perhitungannya sehingga informasi yang didapatkan kurang valid dengan kondisi yang sebenarnya. Komputasi k-MPP juga tidak dapat memproses kueri berbasis interval waktu. Sebagai contoh, pertanyaan yang mungkin diajukan adalah \textit{“$k$-produk apa saja yang paling banyak diminati oleh pelanggan pada bulan Februari hingga September?”}. Dalam hal ini, bulan Februari hingga September disebut dengan interval waktu kueri dan data yang berbasis interval waktu disebut dengan data \textit{time series} atau serial waktu.

Sebagai ilustrasi, produk A adalah produk yang paling banyak diminati oleh pelanggan pada bulan Januari hingga Juni, namun posisinya diungguli oleh produk B yang lebih diminati pelanggan pada bulan Juli hingga September. Pada bulan Oktober, produk B tidak diproduksi lagi karena suatu alasan, sehingga produk A kembali diminati pelanggan.  

Berdasarkan ilustrasi tersebut, produk yang paling unggul berdasarkan kueri k-MPP adalah produk B karena produk B pernah mengungguli produk A walaupun rentang waktu unggulnya lebih pendek daripada produk A. Hal ini terjadi karena komputasi k-MPP hanya mempertimbangkan skor kontribusi pasar yang dihitung dari banyaknya jumlah pelanggan yang lebih menyukai produk tersebut daripada produk lainnya tanpa mempertimbangkan faktor durasi waktu.

Sedangkan jika berdasarkan kueri dengan interval waktu Januari hingga Juli maka produk yang paling unggul adalah produk A; jika berdasarkan kueri dengan interval waktu Juli hingga Agustus maka produk yang paling unggul adalah produk B; jika berdasarkan kueri dengan interval waktu Januari hingga Desember maka produk yang paling unggul adalah produk A karena rentang waktu unggulnya lebih lama daripada produk B.

Tugas Akhir ini bertujuan untuk menjawab permasalahan k-MPP berbasis interval waktu pada data multidimensi dengan serial waktu dengan memodelkan kueri k-MPPTI \textit{(k-Most Promising Products in Time Intervals)} dan merancang kerangka kerja algoritme yang dapat memproses kueri tersebut. Ada tiga jenis algoritme pemrosesan yang dibuat dan dibandingkan: (a) k-MPPTI, yaitu algoritme yang mengadaptasi komputasi k-MPP asli (menggunakan dua tipe kueri \textit{skyline}, yaitu \textit{dynamic skyline} dan \textit{reverse skyline}); (b) k-MPPTI NoRSL, yaitu algoritme yang hanya menggunakan kueri \textit{dynamic skyline} saja; (c) k-MPPTI NoRSL-P, yaitu algoritme k-MPPTI NoRSL yang mengimplementasikan teknik komputasi paralel supaya pemrosesan data menjadi lebih cepat. Efektivitas dan efisiensi ketiga algoritme diuji menggunakan data asli dan sintetis.

\section{Rumusan Masalah}
\tab Rumusan masalah yang diangkat dalam Tugas Akhir ini adalah sebagai berikut:
\begin{enumerate}
	\item Bagaimana desain dan implementasi struktur data dan algoritme untuk \problemm?
	\item Bagaimana kinerja dari struktur data dan algoritme yang dibangun untuk \problemm?
	\item Bagaimana strategi yang optimal untuk meningkatkan efisiensi komputasi \textit{k-Most Promising Products} (k-MPP) berbasis interval waktu pada data multidimensi dengan serial waktu?
\end{enumerate}

\section{Batasan Masalah}
\tab Permasalahan yang dibahas pada Tugas Akhir ini memiliki beberapa batasan sebagai berikut:
\begin{enumerate}
	\item Struktur data dan algoritme dalam komputasi \textit{k-Most Promising Products} (k-MPP) berbasis interval waktu hanya dapat menyimpan dan memproses nilai numerik.
	\item Implementasi struktur data dan algoritme menggunakan bahasa pemrograman Python.
\end{enumerate}

\section{Tujuan}
\tab Tujuan dari Tugas Akhir ini adalah sebagai berikut:

\begin{enumerate}
	\item Merancang dan mengimplementasikan struktur data dan algoritme untuk \problemm.
	\item Mengevaluasi kinerja dari struktur data dan algoritme yang dibangun untuk \problemm.
	\pagebreak
	\item Mengimplementasikan strategi yang optimal untuk meningkatkan efisiensi komputasi \textit{k-Most Promising Products} (k-MPP) berbasis interval waktu pada data multidimensi dengan serial waktu.
\end{enumerate}

\section{Manfaat}
\tab Manfaat yang diharapkan dari penulisan Tugas Akhir ini adalah untuk mengetahui struktur data dan algoritme yang tepat untuk \problemm secara optimal dan efisien. Selain itu, Tugas Akhir ini juga diharapkan dapat memberikan kontribusi pada perkembangan ilmu pengetahuan dan teknologi informasi karena algoritme ini dapat digunakan dalam berbagai hal, khususnya bagi perusahaan untuk membuat bisnisnya menjadi lebih baik dan tepat sasaran.

\section{Metodologi}
\tab Metodologi yang digunakan dalam pengerjaan Tugas Akhir ini adalah sebagai berikut:
\begin{enumerate}
	
	\item Penyusunan proposal Tugas Akhir
	
	\tab Tahap awal untuk memulai pengerjaan Tugas Akhir adalah penyusunan proposal Tugas Akhir yang berisi gagasan untuk \problemm. Proposal ini berisi tentang deskripsi pendahuluan dari Tugas Akhir yang akan dibuat, terdiri atas hal yang menjadi latar belakang diajukannya usulan Tugas Akhir, rumusan masalah yang diangkat, batasan masalah, tujuan, dan manfaat dari pembuatan Tugas Akhir. Selain itu, dijabarkan pula tinjauan pustaka yang digunakan sebagai referensi pendukung pembuatan Tugas Akhir.	
	
	\item Studi literatur
	
	\tab Pada tahap ini dilakukan pencarian informasi dan literatur mengenai metode yang dapat digunakan dalam merancang dan mengimplementasikan struktur data dan algoritme untuk \problemm. Informasi-informasi tersebut bisa didapatkan dari buku, jurnal, maupun internet.
	
	\item Analisis dan perancangan perangkat lunak
	
	\tab Pada tahap ini dilakukan analisis dan perancangan struktur data dan algoritme yang digunakan untuk \problemm berdasarkan literatur yang telah
	dipelajari.
	
	\item Implementasi perangkat lunak
	
	\tab Pada tahap ini dilakukan implementasi atau realiasi dari hasil analisis dan perancangan struktur data dan algoritme yang telah dibuat ke dalam bentuk program.
	
	\item Uji coba dan evaluasi
	
	\tab Pada tahap ini dilakukan uji coba dari struktur data dan algoritme yang telah diimplementasikan. Pengujian akan dilakukan dengan dua cara, yaitu:
	
	\begin{enumerate}
		\item Pengujian waktu eksekusi \textit{(runtime)}
		
		\tab Pengujian yang berfokus pada waktu eksekusi dari struktur data dan algoritme yang dibangun untuk \problemm.
		
		\item Pengujian penggunaan memori \textit{(memory usage)}
		
		\tab Pengujian yang berfokus pada konsumsi memori dari struktur data dan algoritme yang dibangun untuk \problemm.
	\end{enumerate}
	
	\tab Setelah dilakukan uji coba, maka dilakukan evaluasi terhadap kinerja struktur data dan algoritme yang telah diimplementasikan, dengan harapan dapat diperbaiki ke depannya.
	
	\item Penyusunan buku Tugas Akhir
	
	\tab Pada tahap ini dilakukan penyusunan buku Tugas Akhir yang berisi dokumentasi pengerjaan dan laporan hasil pengerjaan Tugas Akhir.
	
\end{enumerate}

\section{Sistematika Penulisan}
Berikut adalah sistematika penulisan buku Tugas Akhir:
\begin{enumerate}
	\item BAB I: PENDAHULUAN
	
	\tab Bab ini berisi latar belakang, rumusan masalah, batasan masalah, tujuan, manfaat, metodologi dan sistematika penulisan Tugas Akhir.
	
	\item BAB II: TINJAUAN PUSTAKA
	
	\tab Bab ini berisi dasar teori mengenai permasalahan dan algoritme penyelesaian yang digunakan dalam Tugas Akhir
	
	\item BAB III: ANALISIS DAN PERANCANGAN SISTEM
	
	\tab Bab ini berisi analisis dan perancangan struktur data dan algoritme yang digunakan dalam penyelesaian permasalahan.
	
	\item BAB IV: IMPLEMENTASI
	
	\tab Bab ini berisi implementasi berdasarkan analisis dan perancangan struktur data dan algortime yang telah dilakukan pada tahap analisis dan perancangan sistem.
	
	\item BAB V: UJI COBA DAN EVALUASI
	
	\tab Bab ini berisi uji coba dan evaluasi dari hasil implementasi yang telah dilakukan pada tahap implementasi.
	
	\pagebreak
	\item BAB VI: PENUTUP
	
	\tab Bab ini berisi kesimpulan dan saran yang didapat dari hasil uji coba dan evaluasi yang telah dilakukan.
	
\end{enumerate}

\cleardoublepage
