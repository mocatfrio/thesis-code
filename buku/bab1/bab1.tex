\chapter{PENDAHULUAN}
\tab Pada bab ini akan dijelaskan latar belakang, rumusan masalah, batasan masalah, tujuan, manfaat, metodologi dan sistematika penulisan Tugas Akhir.

\section{Latar Belakang}
\tab Kemajuan ilmu pengetahuan dan teknologi telah membawa dampak yang cukup besar di bidang bisnis, termasuk mempengaruhi cara produsen dalam melakukan bisnis secara efisien. Produsen dapat mengumpulkan data preferensi pelanggan terhadap produk dan fitur produk dari data penjualan mereka. Selain itu, maraknya penggunaan situs web untuk menjual produk secara \textit{online} juga memungkinkan produsen mendapatkan data preferensi pelanggan terhadap fitur produk produsen lain.\\
\tab Data preferensi pelanggan yang terkumpul dapat dimanfaatkan untuk mendapatkan informasi-informasi penting yang menguntungkan, misalnya untuk mencari produk apa saja yang paling banyak diminati oleh pelanggan sehingga produsen dapat menentukan produk mana yang harus dipilih untuk strategi \textit{targeted marketing} supaya lebih tepat sasaran dan bertahan lama di pasar global.\\
\tab Saat ini, sudah ada komputasi yang dapat menyelesaikan masalah pemilihan
produk dengan memanfaatkan data preferensi pelanggan. \textit{k-Most Promising Products (k-MPP)} \cite{kmpp} adalah sebuah strategi pemilihan produk dengan melakukan pencarian $k$ produk yang paling banyak diminati oleh pelanggan.\\
\tab Komputasi \textit{k-MPP} menggunakan dua tipe kueri \textit{skyline}, yaitu \textit{dynamic skyline} \cite{dynamic-skyline} dan \textit{reverse skyline} \cite{reverse-skyline}. Kueri \textit{dynamic skyline} digunakan untuk mengambil data produk terbaik berdasarkan sudut pandang pelanggan, sedangkan kueri \textit{reverse skyline} digunakan untuk mengambil data pelanggan potensial berdasarkan sudut pandang produsen.\\
\tab Kemudian muncul sebuah pertanyaan, \textit{“Apakah produk yang paling banyak diminati pelanggan akan selalu sama dari waktu ke waktu?”}. Tentu saja para produsen lain tidak akan berdiam diri. Produk-produk baru akan terus bermunculan seiring dengan berjalannya waktu, sehingga produk yang paling diminati pelanggan pun ikut berubah karena adanya produk-produk baru yang dapat mengungguli produk sebelumnya.\\
\tab Sebagai contoh, \textit{smartphone} A adalah produk yang paling banyak diminati oleh pelanggan pada bulan Januari hingga Juni 2018, namun pada bulan Juli posisinya tergeser oleh \textit{smartphone} B yang fitur-fiturnya lebih disukai oleh pelanggan. Dari ilustrasi tersebut, dapat diketahui bahwa interval waktu juga merupakan faktor penting yang harus dipertimbangkan dalam proses pencarian $k$ produk yang paling banyak diminati oleh pelanggan karena sangat berpengaruh terhadap hasil pencarian.\\
\tab Pertanyaan baru yang mungkin akan diajukan oleh produsen atau analis
pemasaran adalah \textit{“$k$ produk apa saja yang paling banyak diminati oleh pelanggan pada bulan Januari hingga Desember 2018?”}. Dalam hal ini, bulan Januari hingga Desember 2018 disebut dengan interval waktu kueri dan data produk yang berbasis interval waktu disebut dengan data \textit{time series} atau serial waktu \cite{time-series}.\\
\tab Untuk menjawab pertanyaan tersebut, dibutuhkan pendekatan lain untuk menyelesaikan kueri \textit{k-Most Promising Products (k-MPP)} berbasis interval waktu pada data multidimensi dengan serial waktu. Algoritme yang dibangun juga mengimplementasikan teknik komputasi paralel supaya pemrosesan data menjadi lebih cepat \cite{kmpp}.

\section{Rumusan Masalah}
\tab Rumusan masalah yang diangkat dalam Tugas Akhir ini adalah sebagai berikut:
\begin{enumerate}
	\item Bagaimana desain dan implementasi struktur data dan algoritme untuk \problem?
	\item Bagaimana kinerja dari struktur data dan algoritme yang dibangun untuk \problem?
	\item Bagaimana strategi yang optimal untuk meningkatkan efisiensi komputasi \textit{k-Most Promising Products (k-MPP)} berbasis interval waktu pada data multidimensi dengan serial waktu?
\end{enumerate}

\section{Batasan Masalah}
\tab Permasalahan yang dibahas pada Tugas Akhir ini memiliki beberapa batasan sebagai berikut:
\begin{enumerate}
	\item Struktur data dan algoritme dalam komputasi \textit{k-Most Promising Products (k-MPP)} hanya dapat menyimpan dan memproses nilai numerik.
	\item Implementasi struktur data dan algoritme menggunakan bahasa pemrograman Python.
	\item \textit{Dataset} yang digunakan adalah data asli dan sintetis.
\end{enumerate}

\section{Tujuan}
\tab Tujuan dari Tugas Akhir ini adalah sebagai berikut:

\begin{enumerate}
	\item Merancang dan mengimplementasikan struktur data dan algoritme untuk \problem.
	\item Mengevaluasi kinerja dari struktur data dan algoritme yang dibangun untuk \problem.
	\item Mengimplementasikan strategi yang optimal untuk meningkatkan efisiensi komputasi \textit{k-Most Promising Products (k-MPP)} berbasis interval waktu pada data multidimensi dengan serial waktu.
\end{enumerate}

\section{Manfaat}
\tab Manfaat yang diharapkan dari penulisan Tugas Akhir ini adalah untuk mengetahui struktur data dan algoritme yang tepat untuk \problem secara optimal dan efisien.\\
\tab Selain itu, Tugas Akhir ini juga diharapkan dapat memberikan kontribusi pada perkembangan ilmu pengetahuan dan teknologi informasi karena algoritme ini dapat digunakan dalam berbagai hal, khususnya bagi produsen untuk membuat bisnisnya menjadi lebih baik dan tepat sasaran.

\section{Metodologi}
\tab Metodologi yang digunakan dalam pengerjaan Tugas Akhir ini adalah sebagai berikut:
\begin{enumerate}
	
	\item Penyusunan proposal Tugas Akhir
	
	\tab Tahap awal untuk memulai pengerjaan Tugas Akhir adalah penyusunan proposal Tugas Akhir yang berisi gagasan untuk \problem. Proposal ini berisi tentang deskripsi pendahuluan dari Tugas Akhir yang akan dibuat, terdiri atas hal yang menjadi latar belakang diajukannya usulan Tugas Akhir, rumusan masalah yang diangkat, batasan masalah, tujuan, dan manfaat dari pembuatan Tugas Akhir. Selain itu, dijabarkan pula tinjauan pustaka yang digunakan sebagai referensi pendukung pembuatan Tugas Akhir.	
	
	\item Studi literatur
	
	\tab Pada tahap ini dilakukan pencarian informasi dan literatur mengenai metode yang dapat digunakan dalam merancang dan mengimplementasikan struktur data dan algoritme untuk \problem. Informasi-informasi tersebut bisa didapatkan dari buku, jurnal, maupun internet.
	
	\item Analisis dan perancangan perangkat lunak
	
	\tab Pada tahap ini dilakukan analisis dan perancangan struktur data dan algoritme yang digunakan untuk \problem berdasarkan literatur yang telah
	dipelajari.
	
	\item Implementasi perangkat lunak
	
	\tab Pada tahap ini dilakukan implementasi atau realiasi dari hasil analisis dan perancangan struktur data dan algoritme yang telah dibuat ke dalam bentuk program.
	
	\item Uji coba dan evaluasi
	
	\tab Pada tahap ini dilakukan uji coba dari struktur data dan algoritme yang telah diimplementasikan. Pengujian akan dilakukan dengan dua cara, yaitu:
	
	\begin{enumerate}
		\item Pengujian waktu eksekusi \textit{(runtime)}
		
		\tab Pengujian yang berfokus pada waktu eksekusi dari struktur data dan algoritme yang dibangun untuk \problem.
		
		\item Pengujian penggunaan memori \textit{(memory usage)}
		
		\tab Pengujian yang berfokus pada konsumsi memori dari struktur data dan algoritme yang dibangun untuk \problem.
	\end{enumerate}
	
	\tab Setelah dilakukan uji coba, maka dilakukan evaluasi terhadap kinerja struktur data dan algoritme yang telah diimplementasikan, dengan harapan dapat diperbaiki ke depannya.
	
	\item Penyusunan buku Tugas Akhir
	
	\tab Pada tahap ini dilakukan penyusunan buku Tugas Akhir yang berisi dokumentasi pengerjaan dan laporan hasil pengerjaan Tugas Akhir.
	
\end{enumerate}

\section{Sistematika Penulisan}
Berikut adalah sistematika penulisan buku Tugas Akhir:
\begin{enumerate}
	\item BAB I: PENDAHULUAN
	
	\tab Bab ini berisi latar belakang, rumusan masalah, batasan masalah, tujuan, manfaat, metodologi dan sistematika penulisan Tugas Akhir.
	
	\item BAB II: TINJAUAN PUSTAKA
	
	\tab Bab ini berisi dasar teori mengenai permasalahan dan algoritme penyelesaian yang digunakan dalam Tugas Akhir
	
	\item BAB III: ANALISIS DAN PERANCANGAN SISTEM
	
	\tab Bab ini berisi analisis dan perancangan struktur data dan algoritme yang digunakan dalam penyelesaian permasalahan.
	
	\item BAB IV: IMPLEMENTASI
	
	\tab Bab ini berisi implementasi berdasarkan analisis dan perancangan struktur data dan algortime yang telah dilakukan pada tahap analisis dan perancangan sistem.
	
	\item BAB V: UJI COBA DAN EVALUASI
	
	\tab Bab ini berisi uji coba dan evaluasi dari hasil implementasi yang telah dilakukan pada tahap implementasi.
	
	\item BAB VI: PENUTUP
	
	\tab Bab ini berisi kesimpulan dan saran yang didapat dari hasil uji coba dan evaluasi yang telah dilakukan.
	
\end{enumerate}

\cleardoublepage
