\chapter{PENDAHULUAN}
\tab Pada bab ini akan dijelaskan latar belakang, rumusan masalah, batasan masalah, tujuan, manfaat, metodologi dan sistematika penulisan Tugas Akhir.

\section{Latar Belakang}
\tab Kemajuan ilmu pengetahuan dan teknologi telah merevolusi cara produsen
dalam melakukan bisnis. Produsen dapat mengumpulkan data preferensi pelanggan
terhadap produk dan fitur produk dari data penjualan mereka. Selain itu, maraknya penggunaan \textit{World Wide Web} untuk menjual produk secara \textit{online} juga memungkinkan produsen mengumpulkan data preferensi pelanggan terhadap fitur produk perusahaan lain.\\
\tab Tak hanya itu, produsen dapat memanfaatkan data preferensi pelanggan tersebut secara cerdas, misalnya untuk mengidentifikasi produk pesaing dan pembeli potensial untuk produk mereka. Produsen juga dapat mencari produk apa saja yang paling banyak diminati oleh pelanggan sehingga ia dapat menentukan produk mana yang harus dipilih untuk strategi \textit{targeted marketing} supaya lebih tepat sasaran dan bertahan di pasar global.\\
\tab Saat ini, sudah ada komputasi yang dapat menyelesaikan masalah pemilihan
produk dengan memanfaatkan data preferensi pelanggan. \textit{k-Most Promising Products (k-MPP)} \cite{kmpp} adalah sebuah strategi pemilihan produk dengan melakukan pencarian $k$ produk yang paling banyak diminati oleh pelanggan.\\
\tab Komputasi \textit{k-MPP} menggunakan dua tipe kueri \textit{skyline}, yaitu \textit{dynamic skyline} \cite{dynamic-skyline} dan \textit{reverse skyline} \cite{reverse-skyline}. Kueri \textit{dynamic skyline} digunakan untuk mengambil data produk berdasarkan sudut pandang pelanggan, sedangkan kueri \textit{reverse skyline} digunakan untuk mengambil data pelanggan berdasarkan sudut pandang produsen.\\
\tab Muncul sebuah pertanyaan, \textit{“Apakah produk yang paling banyak
diminati pelanggan selalu sama dari waktu ke waktu?”}. Tentu saja para produsen tidak akan berdiam diri. Produk-produk baru akan terus bermunculan seiring dengan berjalannya waktu. Begitu pula produk yang paling diminati pelanggan juga ikut berubah karena produk-produk baru tersebut kemungkinan dapat mengungguli produk top sebelumnya.\\
\tab Sebagai contoh, \textit{smartphone} A adalah produk yang paling banyak diminati oleh pelanggan pada bulan Januari hingga Juni 2018, namun pada bulan Juli posisinya tergeser oleh \textit{smartphone} B yang fitur-fiturnya lebih disukai oleh pelanggan. Dari ilustrasi tersebut, dapat diketahui bahwa interval waktu juga merupakan faktor penting yang harus dipertimbangkan dalam proses pencarian $k$ produk yang paling banyak diminati oleh pelanggan karena sangat berpengaruh terhadap hasil pencarian.\\
\tab Pertanyaan baru yang mungkin akan diajukan oleh produsen atau analis
pemasaran adalah \textit{“$k$ produk apa saja yang paling banyak diminati oleh pelanggan pada bulan Januari hingga Desember 2018?”}. Dalam hal ini, bulan Januari hingga Desember 2018 disebut dengan interval waktu kueri dan data produk yang berbasis interval waktu disebut dengan data \textit{time series} atau serial waktu \cite{interval-skyline}.\\
\tab Untuk menjawab pertanyaan tersebut, dibutuhkan penyesuaian dan modifikasi terhadap kueri \textit{k-Most Promising Products (k-MPP)} dan kerangka kerja algoritme pemrosesan kueri yang sudah ada supaya dapat diimplementasikan pada data multidimensi dengan serial waktu. Kerangka kerja tersebut menggunakan struktur data \textit{grid-based index} dan teknik komputasi paralel supaya pemrosesan data dapat dilakukan secara cepat dan akurat \cite{kmpp}.

\section{Rumusan Masalah}
\tab Rumusan masalah yang diangkat dalam Tugas Akhir ini adalah sebagai berikut:
\begin{enumerate}
	\item Bagaimana desain dan implementasi struktur data dan algoritme untuk \problem?
	\item Bagaimana kinerja dari struktur data dan algoritme yang dibangun untuk \problem?
	\item Bagaimana strategi yang optimal untuk meningkatkan efisiensi komputasi \textit{k-Most Promising Products (k-MPP)} berbasis interval waktu pada data multidimensi dengan serial waktu?
\end{enumerate}

\section{Batasan Masalah}
\tab Permasalahan yang dibahas pada Tugas Akhir ini memiliki beberapa batasan, yaitu sebagai berikut:
\begin{enumerate}
	\item Struktur data dan algoritme dalam komputasi \textit{k-Most Promising Products (k-MPP)} hanya dapat menyimpan dan memproses nilai numerik.
	\item Implementasi struktur data dan algoritme menggunakan bahasa pemrograman Python.
	\item \textit{Dataset} yang digunakan adalah data asli dan sintetis.
\end{enumerate}

\section{Tujuan}
\tab Tujuan dari Tugas Akhir ini adalah sebagai berikut:

\begin{enumerate}
	\item Merancang dan mengimplementasikan struktur data dan algoritme untuk \problem.
	\item Mengevaluasi kinerja dari struktur data dan algoritme yang dibangun untuk \problem.
	\item Mengimplementasikan strategi yang optimal untuk meningkatkan efisiensi komputasi \textit{k-Most Promising Products (k-MPP)} berbasis interval waktu pada data multidimensi dengan serial waktu.
\end{enumerate}

\section{Manfaat}
\tab Manfaat yang diharapkan dari penulisan Tugas Akhir ini adalah dapat mendesain dan mengimplementasikan struktur data dan algoritme yang tepat untuk \problem. Tugas Akhir ini juga diharapkan dapat memberikan kontribusi pada perkembangan ilmu pengetahuan dan teknologi informasi.

\section{Metodologi}
Metodologi yang digunakan dalam pengerjaan Tugas Akhir ini adalah sebagai berikut:
\begin{enumerate}
	
	\item Penyusunan proposal Tugas Akhir
	
	Tahap awal untuk memulai pengerjaan Tugas Akhir adalah penyusunan proposal Tugas Akhir yang berisi gagasan untuk menyelesaikan permasalahan \textit{rope} pada studi kasus \problem.
	
	\item Studi literatur
	
	Pada tahap ini dilakukan pencarian informasi dan studi literatur mengenai pengetahuan atau metode yang dapat digunakan dalam penyelesaian masalah. Informasi didapatkan dari materi-materi yang berhubungan dengan algoritma yang digunakan untuk penyelesaian permasalahan ini, materi-materi tersebut didapatkan dari buku, jurnal, maupun internet.
	
	\item Desain
	
	Pada tahap ini dilakukan desain rancangan algoritma yang digunakan dalam solusi untuk pemecahan \problem.
	
	\item Implementasi perangkat lunak
	
	Pada tahap ini dilakukan implementasi atau realiasi dari rancangan desain algoritma yang telah dibangun pada tahap desain ke dalam bentuk program.
	
	\item Uji coba dan evaluasi
	
	Pada tahap ini dilakukan uji coba kebenaran implementasi. Pengujian kebenaran dilakukan pada sistem penilaian daring SPOJ sesuai dengan masalah yang dikerjakan untuk diuji apakah luaran dari program telah sesuai.
	
	\item Penyusunan buku Tugas Akhir
	
	Pada tahap ini dilakukan penyusunan buku Tugas Akhir yang berisi dokumentasi hasil pengerjaan Tugas Akhir.
\end{enumerate}

	\section{Sistematika Penulisan}
	Berikut adalah sistematika penulisan buku Tugas Akhir ini:
	\begin{enumerate}
		\item BABI: PENDAHULUAN
		
		Bab ini berisi latar belakang, rumusan masalah, batasan masalah, tujuan, manfaat, metodologi dan sistematika penulisan Tugas Akhir.
		
		\item BAB II: DASAR TEORI
		
		Bab ini berisi dasar teori mengenai permasalahan dan algoritma penyelesaian yang digunakan dalam Tugas Akhir
		
		\item BAB III: DESAIN
		
		Bab ini berisi desain algoritma dan struktur data yang digunakan dalam penyelesaian permasalahan.
		
		\item BAB IV: IMPLEMENTASI
		
		Bab ini berisi implementasi berdasarkan desain algortima yang telah dilakukan pada tahap desain.
		
		\item BAB V: UJI COBA DAN EVALUASI
		
		Bab ini berisi uji coba dan evaluasi dari hasil implementasi yang telah dilakukan pada tahap implementasi.
		
		\item BAB VI: PENUTUP
		
		Bab ini berisi kesimpulan dan saran yang didapat dari hasil uji coba yang telah dilakukan.
	\end{enumerate}

\cleardoublepage
