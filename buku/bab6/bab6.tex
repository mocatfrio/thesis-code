\chapter{KESIMPULAN}

Pada bab ini dijelaskan mengenai kesimpulan dari hasil uji coba yang telah dilakukan.

\section{Kesimpulan}

Dari hasil uji coba yang telah dilakukan terhadap perancangan dan implementasi algoritma untuk menyelesaikan \problem{} dapat diambil kesimpulan sebagai berikut:

\begin{enumerate}
	\item Implementasi algoritma dengan menggunakan struktur data Rope dapat menyelesaikan \problem{} dengan benar. 
	\item Kompleksitas waktu sebesar $O(\log N)$ dapat menyelesaikan \problem{}.
	\item Waktu yang dibutuhkan program untuk menyelesaikan \problem{} minimum $0.14$ detik, maksimum $0.16$ detik dan rata-rata $0.152$ detik. Memori yang dibutuhkan sebesar $5.0$ MB.
	\item Struktur data Rope yang dibuat sangat baik diaplikasikan untuk melakukan komputasi \textit{string} yang panjang.
\end{enumerate}

\section{Saran}

Pada Tugas Akhir kali ini tentunya terdapat kekurangan serta nilai-nilai yang dapat penulis ambil. Berikut adalah saran-saran yang dapat diambil melalui Tugas Akhir ini:

\begin{enumerate}
	\item Struktur data Rope adalah pendekatan yang sesuai untuk menyelesaikan permasalahan \textit{string} yang sangat panjang.
	\item Struktur data Rope dapat diimplementasikan pada bahasa pemrograman lain.
\end{enumerate}