\chapter{KESIMPULAN DAN SARAN}\label{chap:kesimpulan-saran}

\tab Pada bab ini dijelaskan mengenai kesimpulan dan saran dari hasil uji coba yang telah dilakukan.

\section{Kesimpulan}

\tab Dari proses desain hingga uji coba, dapat diambil beberapa hasil sebagai berikut:

\begin{enumerate}
	\item Tugas akhir ini mengusulkan struktur data grid indeks dan metode $ CSd_\varepsilon $ untuk pengolahan \textit{skyline query} pada \textit{uncertain data streaming} oleh titik bergerak dan objek tidak bergerak. Struktur data grid indeks memecah struktur data graf tradisional menjadi sel-sel yang berisi \textit{node}, \textit{edge}, dan objek. Penyimpanan objek dalam bentuk \textit{SW-Tree} pada setiap \textit{node} membuat proses komputasi lebih cepat.
	
	\item Biaya komputasi pada metode $ CSd_\varepsilon $ jauh lebih baik dibandingkan metode \textit{naive} dari sisi waktu komputasi dan penggunaan memori. Komputasi metode $ CSd_\varepsilon $ lebih cepat 600 kali dibandingkan metode \textit{naive}. Dari sisi penggunaan memori, metode $ CSd_\varepsilon $ lebih hemat 1500 kali dibandingkan metode \textit{naive}.
\end{enumerate}

\section{Saran}

\tab Berikut beberapa saran terkait pengembangan lebih lanjut:

\begin{enumerate}
	\item Pendefinisian jarak $ d_\varepsilon $ dapat dilakukan secara dinamis. Apabila pencarian objek dengan jarak $ d_\varepsilon $ tidak menemukan hasil yang diminta, jarak $ d_\varepsilon $ dapat diperbesar secara dinamis hingga mendapatkan hasil yang sesuai.
	\item Pengembangan algoitma untuk memproses objek \textit{uncertain} yang dapat bergerak secara dinamis.
	\item Pada algoritme ini proses pembaruan \textit{instance} dari \textit{uncertain} objek dilakukan dengan menghapus dan menambahkan objek baru. Hal ini tentunya tidak efisien. Diperlukan algoritme pembaruan objek agar lebih efisien dalam hal waktu komputasi dan penggunaan memori.
\end{enumerate}



