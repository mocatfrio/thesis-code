\chapter{KATA PENGANTAR}
\indent\indent Puji syukur penulis panjatkan kepada Allah Swt. atas pertolongan dan karunia-Nya sehingga penulis dapat menyelesaikan Tugas Akhir yang berjudul:
\begin{center}
	\textbf{\MakeUppercase{\judul}}.
\end{center}

Penelitian Tugas Akhir ini dilakukan untuk memenuhi salah satu syarat meraih gelar Sarjana di Departemen Informatika, Fakultas Teknologi Informasi dan Komunikasi, Institut Teknologi Sepuluh Nopember Surabaya. Dengan selesainya Tugas Akhir ini, diharapkan apa yang telah dikerjakan oleh penulis dapat memberikan manfaat bagi perkembangan ilmu pengetahuan, terutama di bidang teknologi informasi, serta bagi diri penulis sendiri selaku peneliti.

Penulis mengucapkan terima kasih sedalam-dalamnya kepada semua pihak yang telah memberikan dukungan, baik secara langsung maupun tidak langsung, selama penulis mengerjakan Tugas Akhir maupun selama menempuh masa studi antara lain:

\begin{enumerate}
	\item Bapak Darlis Herumurti, S.Kom., M.Kom. selaku Kepala Departemen Informatika ITS, Bapak Radityo Anggoro, S.Kom., M.Sc. selaku koordinator Tugas Akhir, beserta segenap dosen dan karyawan Informatika yang telah memberikan ilmu dan pengalamannya, serta menyediakan berbagai fasilitas dan pelayanan sehingga penulis dapat menempuh studi di Informatika dengan nyaman.
	\item Bapak Bagus Jati Santoso, S.Kom., Ph.D. selaku dosen pembimbing I dan Ibu Henning Titi Ciptaningtyas, S.Kom., M.Kom. selaku dosen pembimbing II yang telah mendampingi penulis sejak penyusunan proposal serta banyak meluangkan waktunya untuk membimbing, memberikan saran dan solusi untuk menyelesaikan Tugas Akhir ini.
	\item Ibu, Bapak, kedua Adik, Akbar dan Alam, serta segenap keluarga yang senantiasa memberikan perhatian, dukungan, pengetahuan, serta kasih sayang yang menjadi semangat dan motivasi bagi diri penulis untuk menyelesaikan Tugas Akhir.
	\item Teman-teman Sempol Bunda: Ajeng, Salma, Napik, Bela, dan Yola, yang telah menemani dan mewarnai masa-masa perkuliahan penulis sejak jaman mahasiswa baru.
	\item Seluruh teman-teman Laboratorium Arsitektur dan Jaringan Komputer (AJK): Mas Syukron, Mas Fatih, Nahda, Satria, Awan, Mas Penyok, Fuad, Didin, Hana, Raldo, Aguel, Khawari, Tamtam, Haura, Lia, Sulton, Mail, Yoga, dan Fawwaz, yang telah menemani, mengganggu, dan membantu penulis selama mengerjakan Tugas Akhir di laboratorium. 
	\item Emak kos terbaik, Mak Ju, atas segala bantuannya selama penulis menempuh studi, dan teman-teman kosan 36, Jakiya, Mutek, Marisa, Mbak Tatak, Alya, Firda, dan Anca.
	\item Teman-teman Penguasa Kosan: Prames, Kikik, Balqis, Tije, Nilam, dan Rini, yang pernah mengajarkan cara bersenang-senang.
	\item Teman-teman \textit{Data Engineers}, Hana dan Rio, sebagai teman seperjuangan dan seperbimbingan Tugas Akhir.
	\item Seluruh teman-teman TC 2015, Mas Andre, dan pihak-pihak lain yang tidak bisa penulis sebutkan satu persatu namanya, yang secara sengaja maupun tidak sengaja turut berkontribusi dalam penyelesaian studi dan Tugas Akhir.
\end{enumerate}

\pagebreak
Penulis mohon maaf apabila masih ada kekurangan pada Tugas Akhir ini. Penulis juga mengharapkan kritik dan saran yang membangun untuk pembelajaran dan perbaikan di kemudian hari. Semoga melalui Tugas Akhir ini Penulis dapat memberikan kontribusi dan manfaat yang sebaik-baiknya. \\ \\ \\

\hfill Surabaya, Juni \tahun \\ \\ \\

\hfill \penulis \\
\cleardoublepage
