\chapter{KATA PENGANTAR}
\indent\indent Puji syukur penulis panjatkan kepada Allah Swt. atas pertolongan dan karunia-Nya sehingga penulis dapat menyelesaikan Tugas Akhir yang berjudul:
\begin{center}
	\textbf{\MakeUppercase{\judul}}.
\end{center}

Penelitian Tugas Akhir ini dilakukan untuk memenuhi salah satu syarat meraih gelar Sarjana di Departemen Informatika, Fakultas Teknologi Informasi dan Komunikasi, Institut Teknologi Sepuluh Nopember Surabaya.

Dengan selesainya Tugas Akhir ini, diharapkan apa yang telah dikerjakan oleh penulis dapat memberikan manfaat bagi perkembangan ilmu pengetahuan, terutama di bidang teknologi informasi, serta bagi diri penulis sendiri selaku peneliti.

Penulis juga mengucapkan banyak terima kasih kepada semua pihak yang telah memberikan dukungan, baik secara langsung maupun tidak langsung, selama penulis mengerjakan Tugas Akhir maupun selama menempuh masa studi antara lain:

\begin{enumerate}
	\item Ibu, bapak, kedua adik, Akbar dan Alam, serta segenap keluarga yang selalu menunggu kedatangan penulis di rumah dan senantiasa memberikan perhatian, dukungan, serta kasih sayang yang menjadi semangat dan motivasi bagi diri penulis.
	\item Bapak Bagus Jati Santoso, S.Kom., Ph.D. selaku dosen pembimbing yang telah banyak meluangkan waktu untuk memberikan ilmu, nasihat, motivasi, pandangan dan bimbingan kepada Penulis baik selama Penulis menempuh masa kuliah maupun selama pengerjaan Tugas Akhir ini.
	\item Ibu Henning Titi Ciptaningtyas, S.Kom., M.Kom. selaku dosen pembimbing yang telah memberikan ilmu, dan masukan kepada Penulis.
	\item Bapak Darlis Herumurti, S.Kom., M.Kom. selaku Kepala Departemen Informatika ITS pada masa pengerjaan Tugas Akhir, Bapak Radityo Anggoro, S.Kom., M.Sc. selaku koordinator Tugas Akhir, dan segenap dosen dan karyawan Informatika yang telah memberikan ilmu, waktu, dan pengalamannya.
	\item Seluruh teman-teman Laboratorium Arsitektur dan Jaringan Komputer angkatan 2015, Nahda, Satria, Hana, Fuad, Didin, Awan, dan Penyok, yang telah menjadi teman berbagi cerita suka dan duka, serta selalu 
\end{enumerate}

Penulis mohon maaf apabila masih ada kekurangan pada Tugas Akhir ini. Penulis juga mengharapkan kritik dan saran yang membangun untuk pembelajaran dan perbaikan di kemudian hari. Semoga melalui Tugas Akhir ini Penulis dapat memberikan kontribusi dan manfaat yang sebaik-baiknya. \\ \\ \\

\hfill Surabaya, Juni \tahun \\ \\ \\

\hfill \penulis \\
\cleardoublepage
